% To create a .eps file, use the command
%      dvips -E -o file.eps figure
% Use gv file.eps to check the bounding box.
% Then to convert the .eps file to a .pdf file while
% preserving the bounding box, use
%         ps2pdf -dEPSCrop file.eps

\documentclass[12pt]{article}

\usepackage{../clp_header}

\thispagestyle{empty}
\input figMac
\def\figdir{}

\begin{document}
\null\vskip1in
%\centerline{\figput{slopeA}\qquad\qquad\figput{slopeB}\qquad\quad}
%\centerline{\figput{slopeC}}
%\centerline{\figput{tangentA}}
%\[
%\hskip-1.3in
%y= f(x_0)+f'(x_0)\,\big(x-x_0\big)\qquad\qquad
%\figplace{tangentB}{0 in}{-1 in}
%\]
%\centerline{\figplace{sinDeriv}{0 in}{0 in}}
%\centerline{\figplace{sinDeriv2}{0 in}{0 in}}
%\centerline{\figplace{sinDeriv5}{0 in}{0 in}\qquad
%\figplace{sinDeriv4}{0 in}{0 in}
%\qquad\figplace{sinDeriv3}{0 in}{0 in}}
%\centerline{\figput{sinDeriv6}}
%\centerline{\figput{CofA2}}
%\centerline{\figput{expGraph}}
%\centerline{\figput{slopeD}}
%\centerline{\figput{tangentSqrt}}
%\centerline{\figput{tangentB}}
%\centerline{\figput{sinDerivL}}
%\centerline{\figput{sinDerivR}}
%\centerline{\figput{sinDeriv6L}}
%\centerline{\figput{sinDeriv6R}}
%\centerline{\figput{sinDeriv6RR}}
%\centerline{\figput{ellipse}}
%\centerline{\figput{ellipseB}}
%\centerline{\figput{astroidB}}
%\centerline{\figput{baseball}}
%\centerline{\figput{bisection}}
%\centerline{\figput{speedA}\figput{speedB}}
%\centerline{\figput{speedA}}
%\centerline{\figput{expInv}}
%\centerline{\figput{sinInv}}
%\centerline{\figput{fInvA}}
%\centerline{\figput{fInvD}}
%\centerline{\figput{sinInvB}}
%\centerline{\figput{cosInvB}}
%\centerline{\figput{tanInvB}}
%\centerline{\figput{triangleAsin}}
%\centerline{\figput{triangleAtan}}
%\centerline{\figput{approx3}}
%\centerline{\figput{triangle45}}
%\centerline{\figput{approx4c}}
%\centerline{\figput{lampShadowM}}
\centerline{\figput{approx4bb}}
%\centerline{\figput{sinInvEg}}
%\centerline{\figput{mvtc}}
%\centerline{\figput{mvtd}}
%\centerline{\figput{mvte}}
%\centerline{\figput{mvtf}}
%\centerline{\figput{concaveUpDown}}
%\centerline{\figput{concaveDown}}



\end{document}