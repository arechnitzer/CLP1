%
% Copyright 2018 Joel Feldman, Andrew Rechnitzer and Elyse Yeager.
% This work is licensed under a Creative Commons Attribution-NonCommercial-ShareAlike 4.0 International License.
% https://creativecommons.org/licenses/by-nc-sa/4.0/
%
\questionheader{ex:s1.6}

%%%%%%%%%%%%%%%%%%
\subsection*{\Conceptual}
%%%%%%%%%%%%%%%%%%

\begin{question}Give an example of a function (you can write a formula, or sketch a graph) that has infinitely many infinite discontinuities.
\end{question}
\begin{hint} Try a repeating pattern.
\end{hint}
\begin{answer} Many answers are possible; the tangent function behaves like this.
\end{answer}
\begin{solution} Many answers are possible; the tangent function behaves like this.
\end{solution}


\begin{question}
When I was born, I was less than one meter tall. Now, I am more than one meter tall.
What is the conclusion of the Intermediate Value Theorem about my height?
\end{question}
\begin{hint} $f$ is my height.
\end{hint}
\begin{answer} At some time between my birth and now, I was exactly one meter tall.
\end{answer}
\begin{solution} If we let $f$ be my height, the time of my birth is $a$, and now is $b$, then we know that $f(a) \leq 1 \leq f(b)$. It is reasonable to assume that my height is a continuous function. So by the IVT, there is some value $c$ between $a$ and $b$ where $f(c)=1$. That is, there is some time (we called it $c$) between my birth and today when I was exactly one meter tall.

Notice the IVT does not say precisely what day I was one meter tall; it only guarantees that such a day occurred between my birth and today.
\end{solution}


\begin{Mquestion}
Give an example (by sketch or formula) of a function $f(x)$, defined on the interval $[0,2]$, with $f(0)=0$, $f(2)=2$, and $f(x)$ never equal to 1. Why does this not contradict the Intermediate Value Theorem?
\end{Mquestion}
\begin{hint} The intermediate value theorem only works for a certain kind of function.
\end{hint}
\begin{answer} One example is $f(x) = \left\{  \begin{array}{ll}
0&\mbox{when }0 \leq x \leq 1\\
2&\mbox{when }1<x \leq 2
\end{array}\right.$. The IVT only guarantees $f(c)=1$ for some $c$ in $[0,2]$ when $f$ is \emph{continuous} over $[0,2]$. If $f$ is not continuous, the IVT says nothing.
\begin{center}
\begin{tikzpicture}
\YEaaxis{.5}{4.5}{.5}{2.5}
\draw[thick] (0,0)node[vertex]{}--(2,0) node[vertex]{};
\draw[thick] (2,2) node[opendot]{}--(4,2) node[vertex]{};
\end{tikzpicture}\end{center}
\end{answer}
\begin{solution} One example is $f(x) = \left\{  \begin{array}{ll}
0&\mbox{when }0 \leq x \leq 1\\
2&\mbox{when }1<x \leq 2
\end{array}\right.$. The IVT only guarantees $f(c)=1$ for some $c$ in $[0,2]$ when $f$ is \emph{continuous} over $[0,2]$. If $f$ is not continuous, the IVT says nothing.
\begin{center}
\begin{tikzpicture}
\YEaaxis{.5}{4.5}{.5}{2.5}
\draw[thick] (0,0)node[vertex]{}--(2,0) node[vertex]{};
\draw[thick] (2,2) node[opendot]{}--(4,2) node[vertex]{};
\end{tikzpicture}\end{center}
\end{solution}


\begin{Mquestion} Is the following a valid statement?
\begin{quote}Suppose $f$ is a continuous function over $[10,20]$, $f(10)=13$, and $f(20)=-13$. Then $f$ has a zero between $x=10$ and $x=20$.\end{quote}
\end{Mquestion}
\begin{answer} Yes
\end{answer}
\begin{solution} Yes. This is a straightforward application of IVT.
\end{solution}

\begin{Mquestion} Is the following a valid statement?
\begin{quote} Suppose $f$ is a continuous function over $[10,20]$, $f(10)=13$, and $f(20)=-13$. Then $f(15)=0$.\end{quote}
\end{Mquestion}
\begin{answer} No
\end{answer}
\begin{solution} No. IVT says that $f(x)=0$ for some $x$ between $10$ and $20$, but it doesn't have to be \emph{exactly half way} between.
\end{solution}


\begin{Mquestion} Is the following a valid statement?
\begin{quote} Suppose $f$ is a function over $[10,20]$, $f(10)=13$, and $f(20)=-13$, and $f$ takes on every value between $-13$ and $13$. Then $f$ is continuous.\end{quote}
\end{Mquestion}
\begin{answer}  No
\end{answer}
\begin{solution} No. IVT says nothing about functions that are not guaranteed to be continuous at the outset. It's quite easy to construct a function that is as described, but not continuous. For example, the function pictured below, whose equation format is somewhat less enlightening than its graph: $f(x)=\left\{\begin{array}{ll}
-\frac{26}{5}x+65,&10 \leq x < 15\\
-\frac{26}{5}x+91,&15 \leq x \leq 20
\end{array}\right.$.
\begin{center}
\begin{tikzpicture}
\YEaaxis{1}{6.5}{3}{3}
\draw (2,.2)--(2,-.2) node[below]{$10$};
\draw (4,.2)--(4,-.2) node[below]{$15$};
\draw (6,.2)--(6,-.2) node[below]{$20$};
\draw (.2,2)--(-.2,2) node[left]{$13$};
\draw (.2,-2)--(-.2,-2) node[left]{$-13$};
\draw[thick] plot[domain=2:4](\x,{-2*\x+6});
\draw[thick] plot[domain=4:6](\x,{-2*\x+10});
\draw (2,2) node[vertex]{};
\draw (4,-2) node[opendot]{};
\draw (4,2) node[vertex]{};
\draw (6,-2) node[vertex]{};
\end{tikzpicture}
\end{center}
\end{solution}


\begin{question}
Suppose $f(t)$ is continuous at $t=5$. True or false: $t=5$ is in the domain of $f(t)$.
\end{question}
\begin{hint}
Compare what is given to you to the definition of continuity.
\end{hint}
\begin{answer}
True.
\end{answer}
\begin{solution}
True. Since $f(t)$ is continuous at $t=5$, that means $\ds\lim_{t \to 5} f(t)=f(5)$. For that to be true, $f(5)$ must exist --- that is, 5 is in the domain of $f(x)$.
\end{solution}



\begin{question}
Suppose $\ds\lim_{t \rightarrow 5}f(t)=17$, and suppose $f(t)$ is continuous at $t=5$. True or false: $f(5)=17$.
\end{question}
\begin{hint}
Compare what is given to you to the definition of continuity.
\end{hint}
\begin{answer}
True.
\end{answer}
\begin{solution}
True. Using the definition of continuity, $\ds\lim_{t \to 5} f(t)=f(5)=17$.
\end{solution}




\begin{question}
Suppose  $\ds\lim_{t \rightarrow 5}f(t)=17$. True or false: $f(5)=17$.
\end{question}
\begin{hint}
What if the function is discontinuous?
\end{hint}
\begin{answer}
In general, false.
\end{answer}
\begin{solution}
In general, false. If $f(t)$ is continuous at $t=5$, then $f(5)=17$; if $f(t)$ is discontinuous at $t=5$, then $f(5)$ either does not exist, or is a number other than 17.

 An example of a function with $\ds\lim_{t \to 5}f(t)=17 \neq f(5)$ is $f(t)=\left\{
\begin{array}{lcr}
17&,&t \neq 5\\
0&,&t=5
\end{array}
 \right.$, shown below.
 \begin{center}
 \begin{tikzpicture}
 \YEaaxis{1}{4}{1}{3}
 \draw[thick] (-1,2)--(4,2);
 \draw (2,2) node[opendot]{};
 \YExcoord{2}{5}
 \draw (2,0) node[vertex]{};
 \end{tikzpicture}
 \end{center}
\end{solution}



\begin{question}
Suppose $f(x)$ and $g(x)$ are continuous at $x=0$, and let $h(x)=\dfrac{xf(x)}{g^2(x)+1}$. What is $\ds\lim_{x \to 0^+} h(x)$?
\end{question}
\begin{hint}
What is $h(0)$?
\end{hint}
\begin{answer}
$\ds\lim_{x \to 0^+} h(x)=0$
\end{answer}
\begin{solution}
Since $f(x)$ and $g(x)$ are continuous at zero, and since $g^2(x)+1$ must be nonzero, then $h(x)$ is continuous at 0 as well. According to the definition of continuity, then $\ds\lim_{x \to 0}h(x)$ exists and is equal to $h(0)=\frac{0f(0)}{g^2(0)+1}=0$.

Since the limit $\ds\lim_{x \to 0}h(x)$ exists and is equal to zero, also the one-sided limit
$\ds\lim_{x \to 0^+}h(x)$ exists and is equal to zero.
\end{solution}


%%%%%%%%%%%%%%%%%%
\subsection*{\Procedural}
%%%%%%%%%%%%%%%%%%

\begin{Mquestion}Find a constant $k$ so that the function \[a(x)=\left\{\begin{array}{ll}
x\sin\left(\frac{1}{x}\right)&\mbox{when } x \neq 0\\
k&\mbox{when }x=0
\end{array}\right.\]
is continuous at $x=0$.
\end{Mquestion}
\begin{hint} Use the definition of continuity.
\end{hint}
\begin{answer}  $k=0$
\end{answer}
\begin{solution} Using the definition of continuity, we need $k=\displaystyle\lim_{x \rightarrow 0} f(x)$. Since the limit is blind to what actually happens to $f(x)$ at $x=0$, this is equivalent to $k=\displaystyle\lim_{x \rightarrow 0} x\sin\left(\frac{1}{x}\right)$. So if we find the limit, we solve the problem.

As $x$ gets small, $\sin\left(\frac{1}{x}\right)$ goes a little crazy (see example~\ref*{eg sinpix}), %1.3.5
  so let's get rid of it by using the Squeeze Theorem. We can bound the function above and below, but we should be a little careful about whether we're going from the left or the right. The reason we need to worry about direction is illustrated with the following observation:
  %%%%%%%%
  If $a\le b$ and $x>0$, then $xa\le xb$. (For example,
           plug in $x=1$, $a=2$, $b=3$.) But if $a\le b$ and $x<0$,
           then $xa \ge xb$. (For example, plug in $x=-1$, $a=2$, $b=3$.)
           So first, let's find
           $\lim\limits_{x\rightarrow 0^-}\sin\left(\frac{1}{x}\right)$.
           When $x<0$,
           \begin{align*}
              1 \ge &\sin\left(\frac{1}{x}\right) \ge -1\\
            \mbox{so: }~~~~
              x(\textcolor{red}{1}) \le x\,&\textcolor{red}{\sin\left(\frac{1}{x}\right) }\le x(\textcolor{red}{-1})
           \end{align*}
and $\displaystyle\lim_{x \rightarrow 0^-} x = \displaystyle\lim_{x \rightarrow 0^-} -x = 0$, so by the Squeeze Theorem, also $\displaystyle\lim_{x \rightarrow 0^-} x\sin\left(\frac{1}{x}\right)=0$.

Now, let's find $\displaystyle\lim_{x \rightarrow 0^+} x\sin\left(\frac{1}{x}\right)$. When $x>0$,
\begin{align*}
 -1 \le &\sin\left(\frac{1}{x}\right) \le 1\\
\mbox{so: }~~~~
x(\textcolor{red}{-1}) \leq x&\textcolor{red}{\sin\left(\frac{1}{x}\right)}\leq x(\textcolor{red}{1})
\end{align*}
and $\displaystyle\lim_{x \rightarrow 0^+} x = \displaystyle\lim_{x \rightarrow 0^+} -x = 0$, so by the Squeeze Theorem, also $\displaystyle\lim_{x \rightarrow 0^+} x\sin\left(\frac{1}{x}\right)=0$.

Since the limits from the left and right agree, we conclude $\displaystyle\lim_{x \rightarrow 0} x\sin\left(\frac{1}{x}\right)=0$, so when $k=0$, the function is continuous at $x=0$.
\end{solution}

\begin{question}Use the Intermediate Value Theorem to show that the function $f(x)=x^3+x^2+x+1$ takes on the value 12345 at least once in its domain.
\end{question}
\begin{hint} If this is your password, you might want to change it.
\end{hint}
\begin{answer} Since $f$ is a polynomial, it is continuous over all real numbers. $f(0)=1<12345$ and $f(12345)=12345^3+12345^2+12345+1>12345$ (since all terms are positive). So by the IVT, $f(c)=12345$ for some $c$ between $0$ and $12345$.
\end{answer}
\begin{solution} Since $f$ is a polynomial, it is continuous over all real numbers. $f(0)=1<12345$ and $f(12345)=12345^3+12345^2+12345+1>12345$ (since all terms are positive). So by the IVT, $f(c)=12345$ for some $c$ between $0$ and $12345$.
\end{solution}

\begin{question}[2015Q]
Describe all points for which the function is continuous: $f(x)=\dfrac{1}{x^2-1}$.
\end{question}
\begin{hint} Find the domain: when is the denominator zero?
\end{hint}
\begin{answer} $(-\infty, -1)\cup (-1,1) \cup (1,+\infty)$
\end{answer}
\begin{solution}
$f(x)$ is a rational function and so is continuous except when
        its denominator is zero. That is, except when $x=1$ and $x=-1$.
\end{solution}


\begin{Mquestion}[2015Q]
 Describe all points for which this function is continuous:
$f(x)=\dfrac{1}{\sqrt{x^2-1}}$.
\end{Mquestion}
\begin{hint}
When is the denominator zero? When is the argument of
         the square root negative?
\end{hint}
\begin{answer}
$(-\infty, -1)\cup (1,+\infty)$
\end{answer}
\begin{solution}
The function is continuous when $x^2-1> 0$, i.e. $(x-1)(x+1)> 0$, which yields the
intervals $(-\infty, -1)\cup (1,+\infty)$.
\end{solution}


\begin{question}[2015Q]
Describe all points for which this function is continuous:
$\dfrac{1}{\sqrt{1+\cos(x)}}$.
\end{question}
\begin{hint} When is the denominator zero? When is the argument of the square root negative?
\end{hint}
\begin{answer}
The function is continuous \emph{except} at
$x=\pm \pi, \pm 3\pi, \pm 5\pi, \dots$.
\end{answer}
\begin{solution}
The function $1/\sqrt{x}$ is continuous on $(0,+\infty)$ and the function  $\cos(x) +
1$ is  continuous everywhere.

So $1/\sqrt{\cos(x) + 1 }$ is continuous except when $\cos x=-1$. This
happens when $x$ is an odd multiple of $\pi$. Hence the function is continuous except at
$x=\pm \pi, \pm 3\pi, \pm 5\pi, \dots$.
\end{solution}


\begin{question}[2015Q]
Describe all points for which this function is continuous: $f(x)=\dfrac{1}{\sin x}$.
\end{question}
\begin{hint}
There are infinitely many points where it is \emph{not} continuous.
\end{hint}
\begin{answer} $x \neq n\pi,$ where $n$ is any integer
\end{answer}
\begin{solution}
The function is continuous when $\sin(x)\neq 0$. That is, when $x$ is not an integer
multiple of $\pi$.
\end{solution}



\begin{question}[2015Q]
Find all values of $c$ such that the following function is continuous at $x=c$:
\[f(x)=\left\{\begin{array}{ccc}
8-cx & \text{if} & x\le c\\
x^2 &  \text{if} & x> c
\end{array}\right.\]
Use the definition of continuity to justify your answer.
\end{question}
\begin{hint} $x=c$ is the important point.
\end{hint}
\begin{answer} $\pm 2$
\end{answer}
\begin{solution}
The function is continuous for $x\ne c$ since each of those two branches are polynomials. So, the only question is whether the function is continuous at $x=c$; for this we need
$$\lim_{x\to c^-}f(x)=f(c)=\lim_{x\to c+}f(x).$$
We compute
$$\lim_{x\to c^-}f(x)=\lim_{x\to c^-}8-cx = 8-c^2;$$
$$f(c)=8-c\cdot c= 8-c^2\text{ and}$$
$$\lim_{x\to c^+}f(x)=\lim_{x\to c^+}x^2=c^2.$$
So, we need $8-c^2=c^2$, which yields $c^2=4$, i.e. $c=-2$ or $c=2$.
\end{solution}


\begin{question}[2015Q]
Find all values of $c$ such that the following function is continuous everywhere:
\begin{align*}
  f(x) &= \begin{cases}
           x^2+c & x\geq 0\\
	  \cos cx & x< 0
          \end{cases}
\end{align*}
Use the definition of continuity to justify your answer.
\end{question}
\begin{hint} The important place is $x=0$.
\end{hint}
\begin{answer} $c=1$
\end{answer}
\begin{solution}
The function is continuous for $x \ne 0$ since $x^2+c$ and
          $\cos cx$ are continuous everywhere. It remains to check continuity at
$x=0$. To do this we must check that the following three are equal.
\begin{align*}
  \lim_{x \to 0^+} f(x) &= \lim_{x\to 0^+} x^2+c = c\\
  f(0) &= c \\
  \lim_{x \to 0^-} f(x) &= \lim_{x\to 0^-} \cos cx = \cos 0 = 1
\end{align*}
Hence when $c=1$ we have the limits agree.
\end{solution}



\begin{Mquestion}[2015Q]
Find all values of $c$ such that the following function is continuous:
\[f(x) = \begin{cases}
      x^2-4 & \text{if } x< c\\
      3x &  \text{if } x \ge c\,.
\end{cases} \]
Use the definition of continuity to justify your answer.
\end{Mquestion}
\begin{hint} The important point is $x=c$.
\end{hint}
\begin{answer}$-1$, $4$
\end{answer}
\begin{solution}
The function is continuous for $x\ne c$ since each of those two branches are
defined by polynomials.  Thus, the only question is
whether the function is continuous at $x=c$. Furthermore,
$$\lim_{x\to c^-}f(x) = c^2-4 $$
and
$$\lim_{x\to c^+}f(x) = f(c) = 3c\,.$$
For continunity we need both limits and the value to agree, so $f$ is
continuous if and only if $c^2-4 = 3c$, that is if and only if
$$ c^2-3c-4 = 0\,.$$
Factoring this as $(c-4)(c+1) = 0$ yields $c=-1$ or $c=+4$.
\end{solution}

\begin{question}[2015Q]
Find all values of $c$ such that the following function is continuous:
\[f(x)=\left\{\begin{array}{ccc}
6-cx & \text{if} & x\le 2c\\
x^2 &  \text{if} & x> 2c
\end{array}\right.\]Use the definition of continuity to justify your answer.
\end{question}
\begin{hint} The important point is $x=2c$.
\end{hint}
\begin{answer} $c=1$, $c=-1$
\end{answer}
\begin{solution} The function is continuous for $x\ne 2c$ since each of those two branches are
polynomials. So, the only question is whether the function is continuous at
$x=2c$; for this we need
$$\lim_{x\to 2c^-}f(x)=f(2c)=\lim_{x\to 2c+}f(x).$$
We compute
$$\lim_{x\to 2c^-}f(x)=\lim_{x\to 2c^-}6-cx = 6-2c^2;$$
$$f(2c)=6-c\cdot 2c= 6-2c^2\text{ and}$$
$$\lim_{x\to 2c^+}f(x)=\lim_{x\to 2c^+}x^2=4c^2.$$
So, we need $6-2c^2=4c^2$, which yields $c^2=1$, i.e. $c=-1$ or $c=1$.
\end{solution}


%%%%%%%%%%%%%%%%%%
\subsection*{\Application}
%%%%%%%%%%%%%%%%%%

\begin{Mquestion}
Show that there exists at least one real number $x$ satisfying $\sin x = x-1$
\end{Mquestion}
\begin{hint}
Consider the function $f(x)=\sin x - x +1$.
\end{hint}
\begin{answer}
This isn't the kind of equality that we can just solve; we'll need a trick, and that trick is the IVT. The general idea is to show that $\sin x$ is somewhere bigger, and somewhere smaller, than $x-1$. However, since the IVT can only show us that a function is equal to a constant, we need to slightly adjust our language. Showing $\sin x = x-1$ is equivalent to showing $\sin x - x + 1 = 0$, so let $f(x)=\sin x - x +1$, and let's show that it has a real root.

First, we need to note that $f(x)$ is continuous (otherwise we can't use the IVT). Now, we need to find a value of $x$ for which it is positive, and for which it's negative. By checking a few values, we find $f(0)$ is positive, and $f(100)$ is negative.  So, by the IVT, there exists a value of $x$ (between $0$ and $100$) for which $f(x)=0$. Therefore, there exists a value of $x$ for which $\sin x = x-1$.
\end{answer}
\begin{solution}
This isn't the kind of equality that we can just solve; we'll need a trick, and that trick is the IVT. The general idea is to show that $\sin x$ is somewhere bigger, and somewhere smaller, than $x-1$. However, since the IVT can only show us that a function is equal to a constant, we need to slightly adjust our language. Showing $\sin x = x-1$ is equivalent to showing $\sin x - x + 1 = 0$, so let $f(x)=\sin x - x +1$, and let's show that it has a real root.

First, we need to note that $f(x)$ is continuous (otherwise we can't use the IVT). Now, we need to find a value of $x$ for which it is positive, and for which it's negative. By checking a few values, we find $f(0)$ is positive, and $f(100)$ is negative.  So, by the IVT, there exists a value of $x$ (between $0$ and $100$) for which $f(x)=0$. Therefore, there exists a value of $x$ for which $\sin x = x-1$.\end{solution}


\begin{question}[2015Q]
Show that there exists at least one real number $c$ such that
$3^c=c^2$.
\end{question}
\begin{hint} Consider the function $f(x)=3^x-x^2$, and how it relates to the problem and the IVT.
\end{hint}
\begin{answer} We let $f(x)=3^x-x^2$. Then $f(x)$ is a continuous function, since both $3^x$ and $x^2$ are continuous for all real numbers.

We want a value $a$ such that $f(a)>0$. We see that $a=0$ works since
$$f(0)=3^0-0=1>0.$$

We want a value $b$ such that $f(b)<0$. We see that
$b=-1$ works since
$$f(-1)=\frac{1}{3}-1<0.$$

So, because $f(x)$ is continuous on $(-\infty, \infty)$ and $f(0)>0$ while
$f(-1)<0$, then the Intermediate Value Theorem guarantees the existence of a
real number $c\in (-1,0)$ such that $f(c)=0$.
\end{answer}
\begin{solution}
We let $f(x)=3^x-x^2$. Then $f(x)$ is a continuous function, since both $3^x$ and $x^2$ are continuous for all real numbers.

We find a value $a$ such that $f(a)>0$. We observe
immediately that $a=0$ works since
$$f(0)=3^0-0=1>0.$$

We find a value $b$ such that $f(b)<0$. We see that
$b=-1$ works since
$$f(-1)=\frac{1}{3}-1<0.$$

So, because $f(x)$ is continuous on $(-\infty, \infty)$ and $f(0)>0$ while
$f(-1)<0$, then the Intermediate Value Theorem guarantees the existence of a
real number $c$ in the interval $(-1,0)$ such that $f(c)=0$.
\end{solution}



\begin{question}[2015Q]
Show that there exists at least one real number $c$ such that $2\tan(c)=c+1$.
\end{question}
\begin{hint} Consider the function $2\tan x - x - 1$ and its roots.
\end{hint}
\begin{answer}
We let $f(x)=2\tan(x)-x-1$. Then $f(x)$ is a continuous function on the interval $(-\pi/2, \pi/2)$ since $\tan(x)=\sin(x)/\cos(x)$ is continuous on this interval, while $x+1$ is a polynomial and therefore continuous for all real numbers.

We find a value $a\in (-\pi/2,\pi/2)$ such that $f(a)<0$. We observe immediately that $a=0$ works since
$$f(0)=2\tan(0)-0-1=0-1=-1<0.$$

We find a value $b\in (-\pi/2, \pi/2)$ such that $f(b)>0$. We see that $b=\pi/4$ works since
$$f(\pi/4)=2\tan(\pi/4) -\pi/4 - 1=2-\pi/4 - 1=1-\pi/4=(4-\pi)/4>0,$$
because $3<\pi<4$.

So, because $f(x)$ is continuous on $[0,\pi/4]$ and $f(0)<0$ while $f(\pi/4)>0$, then the Intermediate Value Theorem guarantees the existence of a real number $c\in (0,\pi/4)$ such that $f(c)=0$.

\end{answer}
\begin{solution}
We let $f(x)=2\tan(x)-x-1$. Then $f(x)$ is a continuous function on the interval $(-\pi/2, \pi/2)$ since $\tan(x)=\sin(x)/\cos(x)$ is continuous on this interval, while $x+1$ is a polynomial and therefore continuous for all real numbers.

We find a value $a\in (-\pi/2,\pi/2)$ such that $f(a)<0$. We observe immediately that $a=0$ works since
$$f(0)=2\tan(0)-0-1=0-1=-1<0.$$

We find a value $b\in (-\pi/2, \pi/2)$ such that $f(b)>0$. We see that $b=\pi/4$ works since
$$f(\pi/4)=2\tan(\pi/4) -\pi/4 - 1=2-\pi/4 - 1=1-\pi/4=(4-\pi)/4>0,$$
because $3<\pi<4$.

So, because $f(x)$ is continuous on $[0,\pi/4]$ and $f(0)<0$ while $f(\pi/4)>0$, then the Intermediate Value Theorem guarantees the existence of a real number $c\in (0,\pi/4)$ such that $f(c)=0$.
\end{solution}



\begin{question}[2015Q]
Show that there exists at least one real number c such that
$\sqrt{\cos(\pi c)} = \sin(2 \pi c) + \frac{1}{2}$.
\end{question}
\begin{hint} Consider the function $f(x) = \sqrt{\cos(\pi x)} - \sin(2\pi x) -1/2$, and be careful about where it is continuous.
\end{hint}
\begin{answer}
Let $f(x) = \sqrt{\cos(\pi x)} - \sin(2\pi x) -1/2$. This function is
continuous provided $\cos(\pi x)\geq 0$. This is  true for $0 \leq x
\leq \frac{1}{2}$.

Now $f$ takes positive values on $[0,1/2]$:
\begin{align*}
  f(0) &= \sqrt{\cos(0)} - \sin(0) -1/2 = \sqrt{1} -1/2 = 1/2.
\end{align*}
And $f$ takes negative values on $[0,1/2]$:
\begin{align*}
  f(1/2) &= \sqrt{\cos(\pi/2)}-\sin(\pi)-1/2 = 0-0-1/2 = -1/2
\end{align*}
(Notice that $f(1/3)=(\sqrt{2}-\sqrt{3})/2-1/2$ also works)

So, because $f(x)$ is continuous on $[0,1/2)$ and $f(0)>0$ while $f(1/2)<0$,
then the  Intermediate Value Theorem guarantees the existence of a real number
$c\in (0,1/2)$ such that $f(c)=0$.
\end{answer}
\begin{solution}
Let $f(x) = \sqrt{\cos(\pi x)} - \sin(2\pi x) -1/2$. This function is
continuous provided $\cos(\pi x)\geq 0$. This is  true for $0 \leq x
\leq \frac{1}{2}$.

Now $f$ takes positive values on $[0,1/2]$:
\begin{align*}
  f(0) &= \sqrt{\cos(0)} - \sin(0) -1/2 = \sqrt{1} -1/2 = 1/2.
\end{align*}
And $f$ takes negative values on $[0,1/2]$:
\begin{align*}
  f(1/2) &= \sqrt{\cos(\pi/2)}-\sin(\pi)-1/2 = 0-0-1/2 = -1/2
\end{align*}
(Notice that $f(1/3)=(\sqrt{2}-\sqrt{3})/2-1/2$ also works)

So, because $f(x)$ is continuous on $[0,1/2)$ and $f(0)>0$ while $f(1/2)<0$,
then the  Intermediate Value Theorem guarantees the existence of a real number
$c\in (0,1/2)$ such that $f(c)=0$.
\end{solution}


\begin{question}[2015Q]
Show that there exists at least one real number $c$ such that
$\dfrac{1}{(\cos\pi c)^2} = c+\dfrac{3}{2}$.
\end{question}
\begin{hint}
Consider the function $f(x)=1/\cos^2(\pi x)-x-\frac{3}{2}$, paying attention to where it is continuous.
\end{hint}
\begin{answer}
We let $f(x)=\dfrac{1}{\cos^2(\pi x)}-x-\dfrac{3}{2}$. Then $f(x)$ is a continuous function on the
interval $(-1/2, 1/2)$ since $\cos x$ is continuous everywhere and non-zero on that
interval.

The function $f$ takes negative values.  For example, when $x=0$:
\begin{align*}
f(0) &= \frac{1}{\cos^2(0)} - 0 - \frac{3}{2} = 1-\frac{3}{2} = -\frac{1}{2} <0.
\end{align*}
It also takes positive values, for instance when $x=1/4$:
\begin{align*}
f(1/4) &= \frac{1}{(\cos \pi/4)^2} - \frac{1}{4} - \frac{3}{2} \\
  &= \frac{1}{1/2} - \frac{1+6}{4} \\
  &= 2 - 7/4 = 1/4 >0.
\end{align*}

By the IVT there is $c$, $0<c<1/4$ such that $f(c)=0$, in which case
\begin{align*}
\dfrac{1}{(\cos\pi c)^2} = c+\dfrac{3}{2}.
\end{align*}
\end{answer}
\begin{solution}
We let $f(x)=\dfrac{1}{\cos^2(\pi x)}-x-\dfrac{3}{2}$. Then $f(x)$ is a continuous function on the
interval $(-1/2, 1/2)$ since $\cos x$ is continuous everywhere and non-zero on that
interval.

The function $f$ takes negative values.  For example, when $x=0$:
\begin{align*}
f(0) &= \frac{1}{\cos^2(0)} - 0 - \frac{3}{2} = 1-\frac{3}{2} = -\frac{1}{2} <0.
\end{align*}
It also takes positive values, for instance when $x=1/4$:
\begin{align*}
f(1/4) &= \frac{1}{(\cos \pi/4)^2} - \frac{1}{4} - \frac{3}{2} \\
  &= \frac{1}{1/2} - \frac{1+6}{4} \\
  &= 2 - 7/4 = 1/4 >0.
\end{align*}

By the IVT there is $c$, $0<c<1/4$ such that $f(c)=0$, in which case
\begin{align*}
\dfrac{1}{(\cos\pi c)^2} = c+\dfrac{3}{2}.
\end{align*}
\end{solution}







\begin{question}Use the intermediate value theorem to find an interval of length one containing a root of $f(x)=x^7-15x^6+9x^2-18x+15$.
\end{question}
\begin{hint} We want $f(x)$ to be 0; 0 is between a positive number and a negative number. Try evaluating $f(x)$ for some integer values of $x$.
\end{hint}
\begin{answer} $[0,1]$ is the easiest answer to find. Also acceptable are $[-2,-1]$ and $[14,15]$.
\end{answer}
\begin{solution} $f(x)$ is a polynomial, so it's continuous everywhere. If we can find values $a$ and $b$ so that $f(a)>0$ and $f(b)<0$, then by the IVT, there will exist some $c$ in $(a,b)$ where $f(c)=0$; that is, there is a root in the interval $[a,b]$. Let's start plugging in easy values of $x$.

$f(0)=15$, and $f(1)=1-15+9-18+15=-8$. Since 0 is between $f(0)$ and $f(1)$, and since $f$ is continuous, by IVT there must be some $x$ in $[0,1]$ for which $f(x)=0$: that is, there is some root in $[0,1]$.

That was the easiest interval to find. If you keep playing around, you find two more. $f(-1)=26$ (positive) and $f(-2)=-1001$ (negative), so there is a root in $[-2,-1]$.

The arithmetic is nasty, but there is also a root in $[14,15]$.

This is an important trick. For high-degree polynomials, it is often impossible to get the exact values of the roots. Using the IVT, we can at least give a range where a root must be.
\end{solution}


\begin{question}Use the intermediate value theorem to give a decimal approximation of $\sqrt[3]{7}$ that is correct to at least two decimal places. You may use a calculator, but only to add, subtract, multiply, and divide.
\end{question}
\begin{hint} $\sqrt[3]{7}$ is the value where $x^3=7$.
\end{hint}
\begin{answer} 1.91
\end{answer}
\begin{solution}
Let $f(x)=x^3$. Since $f$ is a polynomial, it is continuous everywhere. If $f(a)<7<f(b)$, then $\sqrt[3]{7}$ is somewhere between $a$ and $b$. If we can find $a$ and $b$ that satisfy these inequalities, and are very close together, that will give us a good approximation for $\sqrt[3]{7}$.

\begin{itemize}
\item Let's start with integers. $1^3<7<2^3$, so $\sqrt[3]{7}$ is in the interval $(1,2)$.
\item Let's narrow this down, say by testing $f(1.5)$. $(1.5)^3 = 3.375<7$, so $\sqrt[3]{7}$ is in the interval $(1.5,2)$.
\item  Let's narrow further, say by testing $f(1.75)$. $(1.75)^3\approx 5.34<7$, so $\sqrt[3]{7}$ is in the interval $(1.75,2)$.
\item Testing various points, we find $f(1.9)<7<f(2)$, so $\sqrt[3]{7}$ is between $1.9$ and $2$.
\item By testing more, we find
$f(1.91)<7<f(1.92)$, so $\sqrt[3]{7}$ is in $(1.91,1.92)$.
\item In order to get an approximation for $\sqrt[3]{7}$ that is rounded to two decimal places, we have to know whether $\sqrt[3]{7}$ is greater or less than $1.915$; indeed $f(1.915) \approx 7.02>7$, so
$\sqrt[3]{7}<1.915$; then rounded to two decimal places, $\sqrt[3]{7}\approx 1.91$.
\end{itemize}
If this seems like the obvious way to approximate $\sqrt[3]{7}$, that's good. The IVT is a formal statement of a very intuitive principle.
\end{solution}



\begin{question}
Suppose $f(x)$ and $g(x)$ are functions that are continuous over the interval $[a,b]$, with $f(a) \leq g(a)$ and $g(b)\leq f(b)$. Show that there exists some $c \in [a,b]$ with $f(c)=g(c)$.
\end{question}
\begin{hint}
You need to consider separately the cases where $f(a) < g(a)$ and $f(a)=g(a)$. Let $h(x)=f(x)-g(x)$. What is $h(c)$?
\end{hint}
\begin{answer}
\begin{itemize}
\item If $f(a)=g(a)$, or $f(b)=g(b)$, then we simply take $c=a$ or $c=b$.
\item Suppose $f(a) \neq g(a)$ and $f(b) \neq g(b)$. Then $f(a)<g(a)$ and $g(b)<f(b)$, so if we define $h(x)=f(x)-g(x)$, then $h(a)<0$ and $h(b)>0$. Since $h$ is the difference of two functions that are continuous over $[a,b]$, also $h$ is continuous over $[a,b]$. So, by the Intermediate Value Theorem, there exists some $c \in (a,b)$ with $h(c)=0$; that is, $f(c)=g(c)$.
\end{itemize}
\end{answer}
\begin{solution}
\begin{itemize}
\item If $f(a)=g(a)$, or $f(b)=g(b)$, then we simply take $c=a$ or $c=b$.
\item Suppose $f(a) \neq g(a)$ and $f(b) \neq g(b)$. Then $f(a)<g(a)$ and $g(b)<f(b)$, so if we define $h(x)=f(x)-g(x)$, then $h(a)<0$ and $h(b)>0$. Since $h$ is the difference of two functions that are continuous over $[a,b]$, also $h$ is continuous over $[a,b]$. So, by the Intermediate Value Theorem, there exists some $c \in (a,b)$ with $h(c)=0$; that is, $f(c)=g(c)$.
\end{itemize}
\end{solution}
