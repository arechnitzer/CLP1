%
% Copyright 2018 Joel Feldman, Andrew Rechnitzer and Elyse Yeager.
% This work is licensed under a Creative Commons Attribution-NonCommercial-ShareAlike 4.0 International License.
% https://creativecommons.org/licenses/by-nc-sa/4.0/
%
\questionheader{ex:s2.10}
Reminder: in these notes, we use $\log x$ to mean $\log_e x$, which is also commonly written elsewhere as $\ln x$.

%%%%%%%%%%%%%%%%%%
\subsection*{\Conceptual}
%%%%%%%%%%%%%%%%%%

\begin{question}
The volume in decibels (dB) of a sound is given by the formula:
\[V(P)=10\log_{10}\left(\frac{P}{S}\right)\]
where $P$ is the intensity of the sound  and $S$ is the intensity of a standard baseline sound. (That is: $S$ is some constant.)

How much noise will ten speakers make, if each speaker produces 3dB of noise? What about one hundred speakers?
\end{question}
\begin{hint}
Each speaker produces 3dB of noise, so if $P$ is the power of one speaker,
$3=V(P)=10\log_{10}\left(\frac{P}{S}\right)$. Use this to find $V(10P)$ and $V(100P)$.
\end{hint}
\begin{answer}
Ten speakers: 13 dB. One hundred speakers: 23 dB.
\end{answer}
\begin{solution}
We are given that one speaker produces 3dB. So if $P$ is the power of one speaker,
\begin{align*}3=V(P)&=10\log_{10}\left(\frac{P}{S}\right).
\intertext{So, for ten speakers:}
V(10P)&=10\log_{10}\left(\frac{10P}{S}\right)=10\log_{10}\left(\frac{P}{S}\right)+10\log_{10}\left(10\right)\\
&=3+10(1)=13 \mathrm{dB}
\intertext{and for one hundred speakers:}
V(100P)&=10\log_{10}\left(\frac{100P}{S}\right)=10\log_{10}\left(\frac{P}{S}\right)+10\log_{10}\left(100\right)\\
&=3+10(2)=23 \mathrm{dB}
\end{align*}
\end{solution}

\begin{Mquestion}
An investment of \$1000 with an interest rate of 5\% per year grows to
\[A(t)=1000e^{t/20}\]
dollars after $t$ years. When will the investment double?
\end{Mquestion}
\begin{hint}
The question asks you when $A(t)=2000$. So, solve $2000=1000e^{t/20}$ for $t$.
\end{hint}
\begin{answer}
$20\log2 \approx 14$ years
\end{answer}
\begin{solution}
The investment doubles when it hits \$2000. So, we find the value of  $t$
that gives $A(t)=2000$:
\begin{align*}
2000&=A(t)\\
2000&=1000e^{t/20}\\
2&=e^{t/20}\\
\log 2 &=\frac{t}{20}\\
20\log2&=t
\end{align*}
\end{solution}


\begin{Mquestion}
Which of the following expressions, if any, is equivalent to $\log\left(\cos^2 x\right)$?
\[(\mbox{a})~ 2\log(\cos x) \qquad (\mbox{b})~ 2\log|\cos x | \qquad (\mbox{c})~ \log^2(\cos x) \qquad (\mbox{d})~\log(\cos x^2))\]
\end{Mquestion}
\begin{hint}
What happens when $\cos x$ is a negative number?
\end{hint}
\begin{answer}
(b)
\end{answer}
\begin{solution}
From our logarithm rules, we know that when $y$ is \emph{positive}, $\log (y^2)=2\log y$. However, the expression $\cos x$ does not always take on positive values, so (a) is not correct. (For instance, when $x=\pi$, $\log(\cos^2 x)=\log(\cos^2\pi)=\log\left((-1)^2\right) = \log(1)=0$, while $2\log (\cos \pi)=2\log(-1)$, which does not exist.)

Because $\cos^2 x$ is never negative, we notice that $\cos^2 x = |\cos x|^2$. When $\cos x$ is nonzero, $|\cos x|$ is positive, so our logarithm rules tell us $\log\left(|\cos x|^2\right) =2\log|\cos x |$. When $\cos x$ is exactly zero, then both $\log(\cos^2x)$ and $2\log|\cos x|$ do not exist. So, $\log(\cos^2x) = 2\log|\cos x|$.
\end{solution}


%%%%%%%%%%%%%%%%%%
\subsection*{\Procedural}
%%%%%%%%%%%%%%%%%%
\begin{Mquestion}
Differentiate $f(x)=\log(10x)$.
\end{Mquestion}
\begin{hint}
There are two easy ways: use the chain rule, or simplify first.
\end{hint}
\begin{answer}
$f'(x)=\dfrac{1}{x}$
\end{answer}
\begin{solution}
\begin{itemize}
\item Solution 1: Using the chain rule, $\ds\diff{}{x}\left\{\log(10x)\right\}=\dfrac{1}{10x}\cdot 10=\frac{1}{x}$.
\item Solution 2: Simplifying,
$\ds\diff{}{x}\left\{\log(10x)\right\}=\ds\diff{}{x}\left\{\log(10)+\log x\right\} = 0+\frac{1}{x}=\frac{1}{x}$.
\end{itemize}
\end{solution}

\begin{Mquestion}
Differentiate $f(x)=\log(x^2)$.
\end{Mquestion}
\begin{hint}
There are two easy ways: use the chain rule, or simplify first.
\end{hint}
\begin{answer}
$f'(x)=\dfrac{2}{x}$
\end{answer}
\begin{solution}
\begin{itemize}
\item Solution 1: Using the chain rule,
$\ds\diff{}{x}\left\{\log(x^2)\right\}=\frac{1}{x^2}\cdot 2x = \frac{2}{x}$.
\item Solution 2: Simplifying,
$\ds\diff{}{x}\left\{\log(x^2)\right\}= \ds\diff{}{x}\left\{2\log(x)\right\}=   \frac{2}{x}$.
\end{itemize}
\end{solution}

\begin{Mquestion}
Differentiate $f(x)=\log(x^2+x)$.
\end{Mquestion}
\begin{hint}
Don't be fooled by a common mistake: $\log(x^2+x)$ is \emph{not} the same as $\log(x^2)+\log x$.
\end{hint}
\begin{answer}
$f'(x)=\dfrac{2x+1}{x^2+x}$
\end{answer}
\begin{solution}
Don't be fooled by a common mistake: $\log(x^2+x)$ is \emph{not} the same as $\log(x^2)+\log x$. \\
We differentiate using the chain rule:
$\ds\diff{}{x}\left\{\log(x^2+x)\right\}=\dfrac{1}{x^2+x}\cdot(2x+1) = \dfrac{2x+1}{x^2+x}$.
\end{solution}

\begin{Mquestion}
Differentiate $f(x)=\log_{10}x$.
\end{Mquestion}
\begin{hint}
Use the base-change formula to convert this to natural logarithm (base $e$).
\end{hint}
\begin{answer}
$f'(x) = \dfrac{1}{x\log10}$
\end{answer}
\begin{solution}
We know the derivative of the natural logarithm (base $e$), so we use the base-change formula:
\begin{align*}
f(x)=\log_{10}x&=\frac{\log x}{\log 10}
\intertext{Since $\log 10$ is a constant:}
f'(x)&=\frac{1}{x\log 10}.
\end{align*}
\end{solution}


\begin{question}[1997A]
Find the derivative of $y=\dfrac{\log x}{x^3}$.
\end{question}
\begin{answer} $y'=\dfrac{1-3\log x}{x^4}$
\end{answer}
\begin{solution}
\begin{itemize}
\item Solution 1: Using the quotient rule,
\[
y'=\frac{x^3\frac{1}{x}-(\log x)\cdot 3x^2}{x^6}=
\frac{x^2-3x^2\log x}{x^6}=\frac{1-3\log x}{x^4}.\]

\item Solution 2: Using the product rule with
$y=\log x \cdot x^{-3}$,
\[y'=\frac{1}{x}x^{-3}+\log x \cdot(-3)x^{-4}=x^{-4}(1-3\log x)\]
\end{itemize}
\end{solution}



\begin{Mquestion}
Evaluate $\ds\diff{}{\theta} \log(\sec \theta)$.
\end{Mquestion}
\begin{hint}
Use the chain rule.
\end{hint}
\begin{answer}
$\ds\diff{}{\theta} \log(\sec \theta) = \tan \theta$
\end{answer}
\begin{solution}
Using the chain rule,
\begin{align*}
\ds\diff{}{\theta} \log(\sec \theta)&=\frac{1}{\sec \theta}\cdot (\sec \theta \cdot \tan \theta)\\
&=\tan\theta
\end{align*}
Remark: the domain of the function $\log(\sec \theta)$ is those values of $\theta$ for which $\sec\theta$ is positive: so, the intervals $\left(\left(2n-\frac{1}{2}\right)\pi,\left(2n+\frac{1}{2}\right)\pi\right)$ where $n$ is any integer. Certainly the tangent function has a larger domain than this, but outside the domain of $\log(\sec \theta)$, $\tan \theta$ is not the derivative of $\log(\sec \theta)$.
\end{solution}




\begin{question}
Differentiate the function $f(x)=e^{\cos\left(\log x\right)}$.
\end{question}
\begin{hint}  Use the chain rule twice.
\end{hint}
\begin{answer}
$f'(x)=\dfrac{-e^{\cos(\log x)}\sin(\log x)}{x}$
\end{answer}
\begin{solution}
Let's start in with the chain rule.
\begin{align*}
f'(x)
  &= e^{\textcolor{red}{\cos\left(\log x\right)}} \cdot \diff{}{x} \left\{ \textcolor{red}{\cos\left( \log x
\right)}\right\}
\intertext{We'll need the chain rule again:}
  &= e^{\cos\left(\log x\right)} (-\sin(\textcolor{orange}{\log x})) \cdot \diff{}{x}\{\textcolor{orange}{ \log x} \}\\
  &= e^{\cos\left(\log x\right)} (-\sin(\log x)) \cdot \frac{1}{x}\\
  &=\frac{-e^{\cos(\log x)}\sin(\log x)}{x}
\end{align*}
Remark: Although we have a logarithm in the exponent, we can't cancel. The expression $e^{\cos (\log x)}$ is \emph{not} the same as the expression $x^{\cos x}$, or $\cos x$.
\end{solution}


\begin{question}[1996D]
Evaluate the derivative. You do not need to simplify your answer.
\[y=\log(x^2+\sqrt{x^4+1})\]
\end{question}
\begin{hint} You'll need to use the chain rule twice.
\end{hint}
\begin{answer} $y'=\dfrac{2x+\frac{4x^3}{2\sqrt{x^4+1}}}{x^2+\sqrt{x^4+1}}$
\end{answer}
\begin{solution}
\begin{align*}
y&=\log(x^2+\sqrt{x^4+1})
\intertext{So, we'll need the chain rule:}
y'&=\frac{\diff{}{x}\left\{\textcolor{red}{x^2+\sqrt{x^4+1}}\right\}}{\textcolor{red}{x^2+\sqrt{x^4+1}}}\\
&=\frac{2x+\diff{}{x}\left\{\sqrt{x^4+1}\right\}}{x^2+\sqrt{x^4+1}}
\intertext{We need the chain rule again:}
&=\frac{2x+\frac{\diff{}{x}\left\{\textcolor{red}{x^4+1}\right\}}{2\sqrt{\textcolor{red}{x^4+1}}}}{x^2+\sqrt{x^4+1}}\\
&=\frac{2x+\frac{4x^3}{2\sqrt{x^4+1}}}{x^2+\sqrt{x^4+1}}.
\end{align*}
\end{solution}

\begin{question}[2015Q]
 Differentiate $\sqrt{-\log(\cos x)}$.
\end{question}
\begin{hint} Use the chain rule.
\end{hint}
\begin{answer} $\dfrac{\tan x}{2\sqrt{-\log(\cos x)}}$
\end{answer}
\begin{solution}
This requires us to apply the chain rule twice.
\begin{align*}
  \diff{}{x} \left\{ \sqrt{-\log(\cos x)} \right\}
  &= \frac{1}{2\sqrt{\textcolor{red}{-\log(\cos x)}}} \cdot \diff{}{x} \left\{ \textcolor{red}{-\log\left(\cos x\right)} \right\}\\
  &= -\frac{1}{2\sqrt{-\log(\cos x)}} \cdot \frac{1}{\textcolor{orange}{\cos x}} \diff{}{x} \left\{\textcolor{orange}{\cos x} \right\}\\
  &= -\frac{1}{2\sqrt{-\log(\cos x)}} \cdot \frac{1}{\cos x} \cdot \left(-\sin x\right)\\
  &=\frac{\tan x}{2\sqrt{-\log(\cos x)}}
\end{align*}
Remark: it looks strange to see a negative sign in the argument of a square root. Since the cosine function always gives values that are at  most 1, $\log(\cos x)$ is always negative or zero over its domain.
So, $\sqrt{\log(\cos x)}$ is only defined for the points where $\cos x=1$ (and so $\log(\cos x) = 0$--this isn't a very interesting function!
In contrast, $-\log(\cos x)$ is always positive or zero over its domain -- and therefore we can always take its square root.
\end{solution}


\begin{question}[1999H]
Calculate and simplify the derivative of
$\log\big(x+\sqrt{x^2+4}\big)$.
\end{question}
\begin{hint}
Use the chain rule to differentiate.
\end{hint}
\begin{answer} $\dfrac{\sqrt{x^2+4}+x}{x\sqrt{x^2+4}+x^2+4}=
                    \dfrac{1}{\sqrt{x^2+4}}$
\end{answer}
\begin{solution}
Under the chain rule, $\diff{}{x}\log f(x)=\frac{1}{f(x)}f'(x)$. So
\begin{align*}
\diff{}{x}\left\{\log\big(x+\sqrt{x^2+4}\big)\right\}
&=\frac{1}{\textcolor{red}{x+\sqrt{x^2+4}}} \cdot \diff{}{x}\left\{\textcolor{red}{x+\sqrt{x^2+4}}\right\}
\\
&=\frac{1}{x+\sqrt{x^2+4}}\cdot\left(1+\frac{2x}{2\sqrt{x^2+4}}\right)\\
&=\frac{1}{\textcolor{red}{x+\sqrt{x^2+4}}}\cdot
\left(\frac{\textcolor{blue}{2}\textcolor{red}{\sqrt{x^2+4}}
                           +\textcolor{blue}{2}\textcolor{red}{x}}
           {\textcolor{blue}{2}\sqrt{x^2+4}}\right)\\
&=\frac{1}{\sqrt{x^2+4}}
\end{align*}
\end{solution}


\begin{question}[1998H]
Evaluate the derivative of  $g(x)=\log (e^{x^2}+\sqrt{1+x^4})$.
\end{question}
\begin{hint} You can differentiate this by using the chain rule several times.
\end{hint}
\begin{answer}
$g'(x)=\dfrac{2xe^{x^2}\sqrt{1+x^4}+2x^3}{e^{x^2}\sqrt{1+x^4}+1+x^4}$
\end{answer}
\begin{solution}
Using the chain rule,
\begin{align*}
g'(x)&=\frac{\diff{}{x}\{e^{x^2}+\sqrt{1+x^4}\}}{e^{x^2}+\sqrt{1+x^4}}\\
&=\frac{2xe^{x^2}+\frac{4x^3}{2\sqrt{1+x^4}}}{e^{x^2}+\sqrt{1+x^4}}\left(\frac{\sqrt{1+x^4}}{\sqrt{1+x^4}}\right)\\
&=\frac{2xe^{x^2}\sqrt{1+x^4}+2x^3}{e^{x^2}\sqrt{1+x^4}+1+x^4}
\end{align*}
\end{solution}



\begin{question}[1997D]Evaluate the derivative of the following function at $x=1$:
$g(x)=\log\Big(\dfrac{2x-1}{2x+1}\Big)$.
\end{question}
\begin{hint}  Using logarithm rules before you differentiate will make this easier.
\end{hint}
\begin{answer} $\dfrac{4}{3}$
\end{answer}
\begin{solution}
Using logarithm rules makes this an easier problem:
\begin{align*}
g(x) &= \log(2x-1) -\log(2x+1)\\
\mbox{So, }~
g'(x) &= \dfrac{2}{2x-1} -\dfrac{2}{2x+1}\\
\mbox{and }~
g'(1) &= \dfrac{2}{1} -\dfrac{2}{3}=\dfrac{4}{3}
\end{align*}
\end{solution}


\begin{Mquestion}
Evaluate the derivative of the function $f(x) = \log\left(\sqrt{\dfrac{(x^2+5)^3}{x^4+10}}\right)$.
\end{Mquestion}
\begin{hint}   Using logarithm rules before you differentiate will make this easier.
\end{hint}
\begin{answer}
$f'(x)=\dfrac{3x}{x^2+5}-\dfrac{2x^3}{x^4+10}$
\end{answer}
\begin{solution}
We begin by simplifying:
\begin{align*}
f(x) &= \log\left(\sqrt{\dfrac{(x^2+5)^3}{x^4+10}}\right)\\
&=\log\left(\left({\dfrac{(x^2+5)^3}{x^4+10}}\right)^{1/2}\right)\\
&=\frac{1}{2}\log\left(\dfrac{(x^2+5)^3}{x^4+10}\right)\\
&=\frac{1}{2}\left[\log\left({(x^2+5)^3}\right)-\log({x^4+10})\right]
\\
&=\frac{1}{2}\left[3\log\left({(x^2+5)}\right)-\log({x^4+10})\right]
\intertext{Now, we differentiate using the chain rule:}
f'(x)&=\frac{1}{2}\left[ 3\frac{2x}{x^2+5}-\frac{4x^3}{x^4+10} \right]\\
&=\frac{3x}{x^2+5}-\frac{2x^3}{x^4+10}
\end{align*}
Remark: it is a common mistake to write $\log(x^2+4)$ as $\log(x^2)+\log(4)$. These expressions are not equivalent!
\end{solution}


\begin{question}
Evaluate $f'(2)$ if $f(x) = \log\big(g\big(xh(x)\big)\big)$,
$h(2) = 2$, $h'(2) = 3$, $g(4) = 3$, $g'(4) = 5$.
\end{question}
\begin{hint}
First, differentiate using the chain rule and any other necessary rules. Then, plug in $x=2$.
\end{hint}
\begin{answer}
$\dfrac{40}{3}$
\end{answer}
\begin{solution}
We use the chain rule twice, followed by the product rule:
\begin{align*}
f'(x) &=
\frac{1}{\textcolor{red}{g(xh(x))}}\cdot\diff{}{x}\{\textcolor{red}{g(xh(x))}\}\\
&=\frac{1}{g(xh(x))}\cdot g'(\textcolor{orange}{xh(x)})\cdot\diff{}{x}\{\textcolor{orange}{xh(x)}\}
\\ &=\dfrac{1}{g\big(xh(x)\big)}\cdot g'\big(xh(x)\big)\cdot\big[h(x)+xh'(x)\big]
\intertext{In particular, when $x=2$:}
f'(2) &= \dfrac{1}{g\big(2h(2)\big)}\cdot g'\big(2h(2)\big)\cdot \big[h(2)+2h'(2)\big]\\
&= \dfrac{g'(4)}{g(4)}\big[2+2\times 3\big]
= \dfrac{5}{3}\big[2+2\times 3\big]\\
&=\dfrac{40}{3}
\end{align*}
\end{solution}


\begin{Mquestion}[2010H]
Differentiate the function
\[g(x)=\pi^x+x^\pi.\]
\end{Mquestion}
\begin{hint}
In the text, you are given the derivative $\ds\diff{}{x} a^x$, where $a$ is a constant.
\end{hint}
\begin{answer}
$g'(x)=\pi^x\log \pi+\pi x^{\pi -1}$
\end{answer}
\begin{solution}
In the text, we saw that $\ds\diff{}{x}\left\{a^x\right\}=a^x\log a$ for any constant $a$. So, $\ds\diff{}{x}\left\{\pi^x\right\}=\pi^x\log \pi$.

 By the power rule, $\ds\diff{}{x}\left\{x^{\pi}\right\}=\pi x^{\pi-1}$.

 Therefore, $g'(x)=\pi^x\log \pi+\pi x^{\pi-1}$.

 Remark: we had to use two different rules for the two different terms in $g(x)$. Although the functions $\pi^x$ and $x^\pi$ look superficially the same, they behave differently, as do their derivatives. A function of the form $(\mbox{constant})^{x}$ is an exponential function and \emph{not eligible for the power rule}, while a function of the form $x^{\mbox{constant}}$ is exactly the class of function the power rule applies to.
\end{solution}





\begin{Mquestion}\label{s2.10xtox}
Differentiate $f(x)=x^x$.
\end{Mquestion}
\begin{hint} You'll need to use logarithmic differentiation. Set $g(x)=\log(f(x))$, and find $g'(x)$. Then, use that to find $f'(x)$. This is the method used in the text to find $\ds\diff{}{x} a^x$.
\end{hint}
\begin{answer}
$f'(x) = x^x(\log x + 1)$
\end{answer}
\begin{solution}
We have the power rule to tell us the derivative of functions of the form $x^n$, where $n$ is a constant. However, here our exponent is not a constant. Similarly, in this section we learned the derivative of functions of the form $a^x$, where $a$ is a constant, but again, our base is not a constant! Although the result $\ds\diff{}{x} a^x=a^x\log a$ is not what we need, the \emph{method} used to differentiate $a^x$ will tell us the derivative of $x^x$.

We'll set $g(x)=\log(x^x)$, because now we can use logarithm rules to simplify:
\begin{align*}
g(x)=\log(f(x))&=x\log x
\intertext{Now, we can  use the product rule to differentiate the right side, and the chain rule to differentiate $\log(f(x))$:}
g'(x)=\frac{f'(x)}{f(x)}&=\log x +x\frac{1}{x}=\log x +1
\intertext{Finally, we solve for $f'(x)$:}
f'(x)&=f(x)(\log x + 1) = x^x(\log x + 1)
\end{align*}
\end{solution}



\begin{question}[2011H]
Find $f'(x)$ if $f(x) = x^x+\log_{10}x$.
\end{question}
\begin{hint}
Use Question~\ref{s2.10xtox} and the base-change formula, $\log_b(a)=\dfrac{\log a}{\log b}$.
\end{hint}
\begin{answer} $x^x(\log x+1)+\dfrac{1}{x\log 10}$
\end{answer}
\begin{solution}
In Question~\ref{s2.10xtox}, we saw $\ds\diff{}{x}\left\{x^x\right\}=x^x(\log x+1)$. Using
the base-change formula, $\log_{10}(x)=\dfrac{\log x}{\log 10}$. Since $\log_{10}$ is a constant,
\begin{align*}
f'(x)&=\diff{}{x}\left\{x^x+\frac{\log x}{\log 10}\right\}\\
&=x^x(\log x+1)+\frac{1}{x\log 10}
\end{align*}
\end{solution}


\begin{question}
Differentiate $f(x) = \sqrt[4]{\dfrac{(x^4+12)(x^4-x^2+2)}{x^3}}$.
\end{question}
\begin{hint} To make this easier, use logarithmic differentiation.
Set $g(x)=\log(f(x))$, and find $g'(x)$. Then, use that to find $f'(x)$. This is the method used in the text to find $\ds\diff{}{x} a^x$, and again in Question~\ref{s2.10xtox}.
\end{hint}
\begin{answer}
$f'(x)=\dfrac{1}{4}\left(
{\sqrt[4]{\dfrac{(x^4+12)(x^4-x^2+2)}{x^3}}}\right)\left(\dfrac{4x^3}{x^4+12}+\dfrac{4x^3-2x}{x^4-x^2+2}-\dfrac{3}{x}\right)$
\end{answer}
\begin{solution}
Rather than set in with a terrible chain rule problem, we'll use logarithmic differentiation. Instead of differentiating $f(x)$, we differentiate a new function $\log(f(x))$, after simplifying.
\begin{align*}
\log(f(x))&=\log\sqrt[4]{\dfrac{(x^4+12)(x^4-x^2+2)}{x^3}}\\
&=\frac{1}{4}\log\left(\frac{(x^4+12)(x^4-x^2+2)}{x^3}\right)\\
&=\frac{1}{4}\left(\log(x^4+12)+\log(x^4-x^2+2)-3\log x\right)
\intertext{Now that we've simplified, we can efficiently differentiate both sides. It is important to remember that we aren't differentiating $f(x)$ directly--we're differentiating $\log(f(x))$.}
\frac{f'(x)}{f(x)}&=\frac{1}{4}\left(\frac{4x^3}{x^4+12}+\frac{4x^3-2x}{x^4-x^2+2}-\frac{3}{x}\right)
\intertext{Our final step is to solve for $f'(x)$:}
{f'(x)}&=f(x)\frac{1}{4}\left(\frac{4x^3}{x^4+12}+\frac{4x^3-2x}{x^4-x^2+2}-\frac{3}{x}\right)
\\
&=\frac{1}{4}\left(
{\sqrt[4]{\dfrac{(x^4+12)(x^4-x^2+2)}{x^3}}}\right)\left(\frac{4x^3}{x^4+12}+\frac{4x^3-2x}{x^4-x^2+2}-\frac{3}{x}\right)
\end{align*}
It was possible to differentiate this function without logarithms, but the logarithms make it more efficient.
\end{solution}

\begin{Mquestion}
Differentiate $f(x)=(x+1)(x^2+1)^2(x^3+1)^3(x^4+1)^4(x^5+1)^5$.
\end{Mquestion}
\begin{hint} To make this easier, use logarithmic differentiation.
Set $g(x)=\log(f(x))$, and find $g'(x)$. Then, use that to find $f'(x)$. This is the method used in the text to find $\ds\diff{}{x} a^x$, and again in Question~\ref{s2.10xtox}.
\end{hint}
\begin{answer}
$f'(x)=(x+1)(x^2+1)^2(x^3+1)^3(x^4+1)^4(x^5+1)^5
\left[\frac{1}{x+1}+\frac{4x}{x^2+1}
+\frac{9x^2}{x^3+1}+\frac{16x^3}{x^4+1}+\frac{25x^4}{x^5+1}\right]$
\end{answer}
\begin{solution}
It's possible to do this using the product rule a number of times, but it's easier to use logarithmic differentiation. Set
\begin{align*}
g(x)=\log(f(x))&=\log\left[(x+1)(x^2+1)^2(x^3+1)^3(x^4+1)^4(x^5+1)^5\right]
\intertext{Now we can use logarithm rules to change $g(x)$ into a form that is friendlier to differentiate:}
&=\log(x+1)+\log(x^2+1)^2+\log(x^3+1)^3+\log(x^4+1)^4+\log(x^5+1)^5\\
&=\log(x+1)+2\log(x^2+1)+3\log(x^3+1)+4\log(x^4+1)+5\log(x^5+1)
\intertext{Now, we differentiate $g(x)$ using the chain rule:}
g'(x)=\frac{f'(x)}{f(x)}&=\frac{1}{x+1}+\frac{4x}{x^2+1}
+\frac{9x^2}{x^3+1}+\frac{16x^3}{x^4+1}+\frac{25x^4}{x^5+1}
\intertext{Finally, we solve for $f'(x)$:}
f'(x)&=f(x)\left[\frac{1}{x+1}+\frac{4x}{x^2+1}
+\frac{9x^2}{x^3+1}+\frac{16x^3}{x^4+1}+\frac{25x^4}{x^5+1}\right]\\
&=(x+1)(x^2+1)^2(x^3+1)^3(x^4+1)^4(x^5+1)^5\\
&~~~\cdot\left[\frac{1}{x+1}+\frac{4x}{x^2+1}
+\frac{9x^2}{x^3+1}+\frac{16x^3}{x^4+1}+\frac{25x^4}{x^5+1}\right]
\end{align*}
\end{solution}


\begin{question} Differentiate
$f(x) = \left(\dfrac{5x^2+10x+15}{3x^4+4x^3+5}\right)\left(\dfrac{1}{10(x+1)}\right)$.
\end{question}
\begin{hint} It's not going to come out nicely, but there's a better way than blindly applying quotient and product rules, or expanding giant polynomials.
\end{hint}
\begin{answer}
$\left(\dfrac{x^2+2x+3}{3x^4+4x^3+5}\right)\left(\dfrac{1}{x^2+2x+3}-\dfrac{6x^2}{3x^4+4x^3+5}-\dfrac{1}{2(x+1)^2}\right)
$
\end{answer}
\begin{solution}
We could do this with quotient and product rules, but it would be pretty painful. Insteady, let's use a logarithm.
\begin{align*}
f(x) &= \left(\dfrac{5x^2+10x+15}{3x^4+4x^3+5}\right)\left(\dfrac{1}{10(x+1)}\right)
= \left(\dfrac{x^2+2x+3}{3x^4+4x^3+5}\right)\left(\dfrac{1}{2(x+1)}\right)
\\
\log(f(x)) &= \log\left[\left(\dfrac{x^2+2x+3}{3x^4+4x^3+5}\right)\left(\dfrac{1}{2(x+1)}\right)\right]\\
&=\log\left(\dfrac{x^2+2x+3}{3x^4+4x^3+5}\right)
+
\log\left(\dfrac{1}{2(x+1)}\right)
\\&=\log\left({x^2+2x+3}\right)-
\log\left({3x^4+4x^3+5}\right)
-\log(x+1)-\log(2)
\intertext{Now we have a function that we can differentiate more cleanly than our original function.}
\diff{}{x}\left\{\log(f(x))\right\}&=\diff{}{x}\left\{
\log\left({x^2+2x+3}\right)-
\log\left({3x^4+4x^3+5}\right)
-
\log\left({x+1}\right)-
\log\left({2}\right)
\right\}\\
\frac{f'(x)}{f(x)}&=\frac{2x+2}{x^2+2x+3}-\frac{12x^3+12x^2}{3x^4+4x^3+5}-\frac{1}{x+1}\\
&=\frac{2(x+1)}{x^2+2x+3}-\frac{12x^2(x+1)}{3x^4+4x^3+5}-\frac{1}{x+1}
\intertext{Finally, we solve for $f(x)$:}
f'(x)&=f(x)\left(\frac{2(x+1)}{x^2+2x+3}-\frac{12x^2(x+1)}{3x^4+4x^3+5}-\frac{1}{x+1}\right)\\
&= \left(\dfrac{x^2+2x+3}{3x^4+4x^3+5}\right)\left(\dfrac{1}{2(x+1)}\right)\left(\frac{2(x+1)}{x^2+2x+3}-\frac{12x^2(x+1)}{3x^4+4x^3+5}-\frac{1}{x+1}\right)\\
&= \left(\dfrac{x^2+2x+3}{3x^4+4x^3+5}\right)\left(\frac{1}{x^2+2x+3}-\frac{6x^2}{3x^4+4x^3+5}-\frac{1}{2(x+1)^2}\right)
\end{align*}
\end{solution}




\begin{question}[2007H]\label{s2.10xtox2} Let $f(x) = (\cos x)^{\sin x}$,
with domain $0<x<\tfrac{\pi}{2}$. Find $f'(x)$.
\end{question}
\begin{hint}
You'll need to use logarithmic differentiation. Set $g(x)=\log(f(x))$, and find $g'(x)$. Then, use that to find $f'(x)$. This is the method used in the text to find $\ds\diff{}{x} a^x$, and again in Question~\eqref{s2.10xtox}.
\end{hint}
\begin{answer}
$f'(x)=(\cos x)^{\sin x}\left[(\cos x) \log (\cos x) - \sin x \tan x\right]$
\end{answer}
\begin{solution}
Since $f(x)$ has the form of a function raised to a functional power, we will use logarithmic differentiation.
\begin{align*}
\log(f(x))&=\log\left( (\cos x)^{\sin x}\right)=\sin x \cdot \log (\cos x)
\intertext{Logarithm rules allowed us to simplify. Now, we differentiate both sides of this equation:}
\frac{f'(x)}{f(x)}&=(\cos x ) \log(\cos x)+ \sin x \cdot \frac{-\sin x}{\cos x}\\
&=(\cos x) \log (\cos x) - \sin x \tan x
\intertext{Finally, we solve for $f'(x)$:}
f'(x)&=f(x)\left[(\cos x) \log (\cos x) - \sin x \tan x\right]\\
&= (\cos x)^{\sin x}\left[(\cos x) \log (\cos x) - \sin x \tan x\right]
\end{align*}

Remark: negative numbers behave in a complicated manner when they are the base of an exponential expression. For example, the expression $(-1)^x$ is defined when $x$ is the reciprocal of an odd number (like $x=\frac{1}{5}$ or $x=\frac{1}{7}$), but not when  $x$ is the reciprocal of an even number (like $x=\frac{1}{2}$). Since the domain of $f(x)$ was restricted to $(0,\tfrac{\pi}{2})$, $\cos x$ is always positive, and we avoid these complications.
\end{solution}


\begin{question}[2006H]\label{s2.10xtox3}
 Find the derivative of $(\tan(x))^x$, when $x$ is in the interval $(0,\pi/2)$.
\end{question}
\begin{hint}
You'll need to use logarithmic differentiation. Set $g(x)=\log(f(x))$, and find $g'(x)$. Then, use that to find $f'(x)$. This is the method used in the text to find $\ds\diff{}{x} a^x$, and again in Question~\eqref{s2.10xtox}.
\end{hint}
\begin{answer}
${\ds\diff{}{x}\left\{(\tan x)^x\right\}}={(\tan x)^x}\left(\log(\tan x) + \dfrac{x}{\sin x \cos x}\right)$
\end{answer}
\begin{solution}
Since $f(x)$ has the form of a function raised to a functional power, we will use logarithmic differentiation. We take the logarithm of the function, and make use of logarithm rules:
\begin{align*}
\log\left((\tan x)^x\right)&=x\log(\tan x)
\intertext{Now, we can differentiate:}
\frac{\diff{}{x}\left\{(\tan x)^x\right\}}{(\tan x)^x}&=\log(\tan x) + x\cdot\frac{\sec^2 x}{\tan x}\\
&=\log(\tan x) + \frac{x}{\sin x \cos x}
\intertext{Finally, we solve for the derivative we want, $\ds\diff{}{x}\{(\tan x)^x\}$:}
{\diff{}{x}\left\{(\tan x)^x\right\}}&={(\tan x)^x}\left(\log(\tan x) + \frac{x}{\sin x \cos x}\right)
\end{align*}

Remark: the restricted domain $(0,\pi/2)$ ensures that $\tan x$ is a positive number, so we avoid the problems that arise by raising a negative number to a variety of powers.
\end{solution}





\begin{question}[2015Q]\label{s2.10xtox4} Find $f'(x)$ if $f(x)= (x^2+1)^{(x^2+1)}$
\end{question}
\begin{hint}
You'll need to use logarithmic differentiation. Set $g(x)=\log(f(x))$, and find $g'(x)$. Then, use that to find $f'(x)$. This is the method used in the text to find $\ds\diff{}{x} a^x$, and again in Question~\eqref{s2.10xtox}.
\end{hint}
\begin{answer} $2x(x^2+1)^{x^2+1} (1+\log(x^2+1))$
\end{answer}
\begin{solution} We use logarithmic differentiation.
\begin{align*}
  \log f(x) &= \log(x^2+1) \cdot (x^2+1)
\intertext{We differentiate both sides to obtain:}
  \dfrac{f'(x)}{f(x)} &= \diff{}{x} \left\{ \log(x^2+1) \cdot (x^2+1) \right\}\\
  &= \frac{2x}{x^2+1}(x^2+1)+2x\log(x^2+1)\\&=2x(1+\log(x^2+1))
\intertext{Now, we solve for $f'(x)$:}
  f'(x) &= f(x) \cdot 2x(1+\log(x^2+1))
 \\
&= (x^2+1)^{x^2+1} \cdot  2x(1+\log(x^2+1))
\end{align*}
\end{solution}




\begin{Mquestion}[2015Q]\label{s2.10xtox5}
Differentiate $f(x)= (x^2+1)^{\sin(x)}$.
\end{Mquestion}
\begin{hint}
You'll need to use logarithmic differentiation. Differentiate $\log(f(x))$, then solve for $f'(x)$. This is the method used in the text to find $\ds\diff{}{x} a^x$.
\end{hint}
\begin{answer}
$f'(x)= (x^2+1)^{\sin(x)} \cdot \left( \cos x \cdot \log(x^2+1) + \frac{2x\sin x}{x^2+1}
\right)$
\end{answer}
\begin{solution}
We use logarithmic differentiation: we modify our function to consider
\begin{align*}
  \log f(x) &= \log(x^2+1) \cdot \sin x
\intertext{We differentiate using the product and chain rules:}
  \dfrac{f'(x)}{f(x)} &= \diff{}{x} \left\{ \log(x^2+1) \cdot \sin x \right\}
  = \cos x \cdot \log(x^2+1) + \frac{2x\sin x}{x^2+1}
\intertext{Finally, we solve for $f'(x)$}
  f'(x) &= f(x) \cdot \left( \cos x \cdot \log(x^2+1) + \frac{2x\sin x}{x^2+1}
\right)  \\
&= (x^2+1)^{\sin(x)} \cdot \left( \cos x \cdot \log(x^2+1) + \frac{2x\sin x}{x^2+1}
\right)
\end{align*}
\end{solution}


\begin{question}[2015Q]\label{s2.10xtox6}
Let $f(x)= x^{\cos^3(x)}$, with domain $(0,\infty)$. Find $f'(x)$.
\end{question}
\begin{hint} You'll need to use logarithmic differentiation. Differentiate $\log(f(x))$, then solve for $f'(x)$. This is the method used in the text to find $\ds\diff{}{x} a^x$.
\end{hint}
\begin{answer}
$x^{\cos^3(x)} \cdot \left( -3\cos^2(x)\sin(x) \log(x) + \dfrac{\cos^3(x)}{x} \right)$
\end{answer}
\begin{solution}
We use logarithmic differentiation; so we modify our function to consider
\begin{align*}
  \log f(x) &= \log(x) \cdot \cos^3(x)
\intertext{Differentiating, we find:}
  \dfrac{f'(x)}{f(x)} &= \diff{}{x} \left\{ \log(x) \cdot \cos^3(x) \right\}
  = 3\cos^2(x)\cdot (-\sin(x)) \cdot \log(x) + \frac{\cos^3(x)}{x}
\intertext{Finally, we solve for $f'(x)$:}
  f'(x) &= f(x) \cdot \left(  -3\cos^2(x)\sin(x)  \log(x) + \frac{\cos^3(x)}{x}
\right) \\
&= x^{\cos^3(x)} \cdot \left( -3\cos^2(x)\sin(x) \log(x) + \frac{\cos^3(x)}{x}
\right)
\end{align*}


Remark: negative numbers behave in a complicated manner when they are the base of an exponential expression. For example, the expression $(-1)^x$ is defined when $x$ is the reciprocal of an odd number (like $x=\frac{1}{5}$ or $x=\frac{1}{7}$), but not when  $x$ is the reciprocal of an even number (like $x=\frac{1}{2}$). Since the domain of $f(x)$ was restricted so that $x$ is always positive, we avoid these complications.
\end{solution}

\begin{question}[2015Q]\label{s2.10xtox7}
Differentiate $f(x)= (3+\sin(x))^{x^2-3}$.
\end{question}
\begin{hint} You'll need to use logarithmic differentiation. Differentiate $\log(f(x))$, then solve for $f'(x)$. This is the method used in the text to find $\ds\diff{}{x} a^x$.
\end{hint}
\begin{answer}
$(3+\sin(x))^{x^2-3}\cdot \left[ 2x\log(3+\sin(x)) + \dfrac{(x^2-3)\cos(x)}{3+\sin(x)}\right]$
\end{answer}
\begin{solution}
We use logarithmic differentiation. So, we modify our function and consider
\begin{align*}
  \log f(x) &= (x^2-3)\cdot \log(3+\sin(x))\,.
\intertext{We differentiate:}
  \frac{f'(x)}{f(x)} &=  \diff{}{x} \left\{(x^2-3)\cdot \log(3+\sin(x)) \right\}\\
  &=2x\log(3+\sin(x)) + (x^2-3)\frac{\cos(x)}{3+\sin(x)}
  \intertext{Finally, we solve for $f'(x)$:}
  f'(x)&= f(x)\cdot
   \left[ 2x\log(3+\sin(x)) + \frac{(x^2-3)\cos(x)}{3+\sin(x)}\right]\\
    &= (3+\sin(x))^{x^2-3}\cdot
   \left[ 2x\log(3+\sin(x)) + \frac{(x^2-3)\cos(x)}{3+\sin(x)}\right]
\end{align*}
\end{solution}
%%%%%%%%%%%%%%%%%%
\subsection*{\Application}
%%%%%%%%%%%%%%%%%%

\begin{question}
Let $f(x)$ and $g(x)$ be differentiable functions, with $f(x)>0$. Evaluate $\ds\diff{}{x}\left\{[f(x)]^{g(x)}\right\}$.
\end{question}
\begin{hint}
Evaluate $\ds\diff{}{x}\left\{\log\left(\left[f(x)\right]^{g(x)}\right)\right\}$.
\end{hint}
\begin{answer}
 $\ds\diff{}{x}\left\{[f(x)]^{g(x)}\right\}=\left[f(x)\right]^{g(x)}\left[
g'(x)\log(f(x))+ \dfrac{g(x)f'(x)}{f(x)}
 \right]$
\end{answer}
\begin{solution}
We will use logarithmic differentiation. First, we take the logarithm of our function, so we can use logarithm rules.
\begin{align*}
\log\left([f(x)]^{g(x)}\right)&=g(x)\log(f(x))
\intertext{Now, we differentiate. On the left side we use the chain rule, and on the right side we use product and chain rules.}
\ds\diff{}{x}\left\{\log\left([f(x)]^{g(x)}\right)\right\}&=\ds\diff{}{x}\left\{g(x)\log(f(x))\right\}\\
\frac{\diff{}{x}\{[f(x)]^{g(x)}\}}{[f(x)]^{g(x)}}&=
g'(x)\log(f(x))+g(x)\cdot\frac{f'(x)}{f(x)}
\intertext{Finally, we solve for the derivative of our original function.}
{\diff{}{x}\{[f(x)]^{g(x)}\}}&={[f(x)]^{g(x)}}\left(
g'(x)\log(f(x))+g(x)\cdot\frac{f'(x)}{f(x)}\right)
\end{align*}

Remark: in this section, we have differentiated problems of this type several times--for example, %Question~\ref{s2.10xtox}, and
Questions \ref{s2.10xtox2}
%\ref{s2.10xtox3},
%\ref{s2.10xtox4},
%\ref{s2.10xtox5},
%\ref{s2.10xtox6}, and
through \ref{s2.10xtox7}.
\end{solution}




\begin{Mquestion}
Let $f(x)$ be a function whose range includes only positive numbers. Show that the curves $y=f(x)$ and $y=\log(f(x))$ have horizontal tangent lines at the same values of $x$.
\end{Mquestion}
\begin{hint}
Differentiate $y=\log(f(x))$. When is the derivative equal to zero?
\end{hint}
\begin{answer}
Let $g(x):=\log(f(x))$. Notice $g'(x)=\frac{f'(x)}{f(x)}$.\\

In order to show that the two curves have horizontal tangent lines at the same values of $x$, we will show two things: first, that if $f(x)$ has a horizontal tangent line at some
value of $x$, then also $g(x)$ has a horizontal tangent line at that value of $x$.
Second, we will show that if $g(x)$ has a horizontal tangent line at some
value of $x$, then also $f(x)$ has a horizontal tangent line at that value of $x$.

Suppose $f(x)$ has a horizontal tangent line where $x=x_0$ for some point $x_0$. This means $f'(x_0)=0$. Then $g'(x_0)=\frac{f'(x_0)}{f(x_0)}$. Since $f(x_0) \neq 0$, $\frac{f'(x_0)}{f(x_0)}=\frac{0}{f(x_0)}=0$, so $g(x)$ also has a horizontal tangent line when $x=x_0$. This shows that whenever $f$ has a horizontal tangent line, $g$ has one too.

Now suppose $g(x)$ has a horizontal tangent line where $x=x_0$ for some point $x_0$. This means $g'(x_0)=0$. Then $g'(x_0)=\frac{f'(x_0)}{f(x_0)}=0$,
so $f'(x_0)$ exists and is equal to zero.
Therefore, $f(x)$ also has a horizontal tangent line when $x=x_0$. This shows that whenever $g$ has a horizontal tangent line, $f$ has one too.
\end{answer}
\begin{solution}
Let $g(x):=\log(f(x))$. Notice $g'(x)=\frac{f'(x)}{f(x)}$.\\

In order to show that the two curves have horizontal tangent lines at the same values of $x$, we will show two things: first, that if $f(x)$ has a horizontal tangent line at some
value of $x$, then also $g(x)$ has a horizontal tangent line at that value of $x$.
Second, we will show that if $g(x)$ has a horizontal tangent line at some
value of $x$, then also $f(x)$ has a horizontal tangent line at that value of $x$.

Suppose $f(x)$ has a horizontal tangent line where $x=x_0$ for some point $x_0$. This means $f'(x_0)=0$. Then $g'(x_0)=\frac{f'(x_0)}{f(x_0)}$. Since $f(x_0) \neq 0$, $\frac{f'(x_0)}{f(x_0)}=\frac{0}{f(x_0)}=0$, so $g(x)$ also has a horizontal tangent line when $x=x_0$. This shows that whenever $f$ has a horizontal tangent line, $g$ has one too.

Now suppose $g(x)$ has a horizontal tangent line where $x=x_0$ for some point $x_0$. This means $g'(x_0)=0$. Then $g'(x_0)=\frac{f'(x_0)}{f(x_0)}=0$,
so $f'(x_0)$ exists and is equal to zero.
Therefore, $f(x)$ also has a horizontal tangent line when $x=x_0$. This shows that whenever $g$ has a horizontal tangent line, $f$ has one too.

Remark: if we were not told that $f(x)$ gives only positive numbers, it would not necessarily be true that $f(x)$ and $\log(f(x))$ have horizontal tangent lines at the same values of $x$. If $f(x)$ had a horizontal tangent line at an $x$-value where $f(x)$ were negative, then $\log(f(x))$ would not exist there, let alone have a horizontal tangent line.
\end{solution}
