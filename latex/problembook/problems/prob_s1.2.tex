%
% Copyright 2018 Joel Feldman, Andrew Rechnitzer and Elyse Yeager.
% This work is licensed under a Creative Commons Attribution-NonCommercial-ShareAlike 4.0 International License.
% https://creativecommons.org/licenses/by-nc-sa/4.0/
%
\questionheader{ex:s1.2}

%%%%%%%%%%%%%%%%%%
\subsection*{\Conceptual}
%%%%%%%%%%%%%%%%%%

\begin{question}
As they are used in this section, what is the difference between speed and velocity?
\end{question}
\begin{answer}
Speed is nonnegative; velocity has a sign (positive or negative) that indicates direction.
\end{answer}
\begin{solution}
Speed is nonnegative; velocity has a sign (positive or negative) that indicates direction.
\end{solution}

\begin{question} Speed can never be negative; can it be zero?
\end{question}
\begin{answer}
Yes--an object that is not moving has speed 0.
\end{answer}
\begin{solution}
Yes--an object that is not moving has speed 0.
\end{solution}


\begin{Mquestion}Suppose you wake up in the morning in your room, then you walk two kilometres to school, walk another two kilometres to lunch, walk four kilometres to a coffee shop to study, then return to your room until the next morning. In the 24 hours from morning to morning, what was your average velocity? (In CLP-1, we are considering functions of one variable. So, at this stage, think of our whole world as being contained in the $x$-axis.)
\end{Mquestion}
\begin{hint}Where did you start, and where did you end?
\end{hint}
\begin{answer}0 kph
\end{answer}
\begin{solution}Since you started and ended in the same place, your difference in position was 0, and  your difference in time was 24 hours. So, your average velocity was $\dfrac{0}{24}=0$ kph.
\end{solution}



\begin{question}
Suppose you drop an object, and it falls for a few seconds. Which is larger: its speed at the one second mark, or its average speed from the zero second mark to the one second mark?
\end{question}
\begin{hint}
Is the object falling faster and faster, slower and slower, or at a constant rate?
\end{hint}
\begin{answer}
The speed at the one second mark is larger than the average speed.
\end{answer}
\begin{solution}
Objects accelerate as they fall --- their speed gets bigger and bigger. 
So in the entire first second of falling, the object is at its fastest at 
the one-second mark. We have defined the average speed over a time interval
to be $\frac{\text{distance moved}}{\text{time taken}}$. We do not yet know 
how to compute the distanced travelled by an object whose speed is not constant. 
But our intuition certainly says that the distance travelled is no more than
the maximum speed times the time taken, so that the average speed is no 
larger that the maximum speed. In the next CLP-1 chapter we will verify that intuition mathematically. See Example~\ref{eg:mvtg} in the CLP-1 text.
So the average speed will be a smaller number then the speed at the 
one-second mark, which is the maximum speed.
\end{solution}

\begin{Mquestion}
The position of an object at time $t$ is given by $s(t)$. Then its average velocity over the time interval $t=a$ to $t=b$ is given by $\dfrac{s(b)-s(a)}{b-a}$. Explain why this fraction also gives the slope of the secant line of the curve $y=s(t)$ from the point $t=a$ to the point $t=b$.
\end{Mquestion}
\begin{hint}
Slope is change in vertical component over change in horizontal component.
\end{hint}
\begin{answer}
The slope of a curve is given by $\dfrac{\mbox{change in vertical component}}{\mbox{change in horizontal component}}$. The change in the vertical component is exactly $s(b)-s(a)$, and the change in the horizontal component is exactly $b-a$.
\end{answer}
\begin{solution}
The slope of a curve is given by $\dfrac{\mbox{change in vertical component}}{\mbox{change in horizontal component}}$. The change in the vertical component is exactly $s(b)-s(a)$, and the change in the horizontal component is exactly $b-a$.
\end{solution}

\begin{Mquestion}Below is the graph of the position of an object at time $t$. For what periods of time is the object's velocity positive?
\begin{center}
\begin{tikzpicture}
\YEtaxis{1}{8}{1}{4}
\foreach \x in {1,...,7}{\YExcoord{\x}{\x}}
\draw[thick] plot[smooth] coordinates{(0,1) (2,3) (4,1) (6,0) (7,1)} node[right]{$y=s(t)$};
\end{tikzpicture}
\end{center}
\end{Mquestion}
\begin{hint}
Sign of velocity gives direction of motion:
the velocity is positive at time $t$ if $s(t)$ is
            increasing at time $t$.
\end{hint}
\begin{answer}
$(0,2) \cup (6,7)$
\end{answer}
\begin{solution}
The velocity is positive when the object is going in the increasing direction; it is going ``up" on the graph when $t$ is between 0 and 2, and when $t$ is between $6$ and $7$. So, the velocity is positive when $t$ is in $(0,2) \cup (6,7)$.
\end{solution}




%%%%%%%%%%%%%%%%%%%
\subsection*{\Procedural}
%%%%%%%%%%%%%%%%%%%
\begin{question}
Suppose the position of a body at time $t$ (measured in seconds) is given by
$s(t)=3t^2+5$.
\begin{enumerate}[(a)]
\item What is the average velocity of the object from 3 seconds to 5 seconds?
\item What is the velocity of the object at time $t=1$?
\end{enumerate}
\end{question}
\begin{hint}
Velocity is distance over time.
\end{hint}
\begin{answer}
\begin{enumerate}[(a)]
\item 24 units per second.
\item 6 units per second
\end{enumerate}
\end{answer}
\begin{solution}
\begin{enumerate}[(a)]
\item Average velocity: $\dfrac{\mbox{change in position}}{\mbox{change in time}} = \dfrac{s(5)-s(3)}{5-3} = \dfrac{(3\cdot 5^2+5)-(3\cdot 3^2+5)}{5-3}=24$ units per second.
\item From the notes, we know the velocity of an object at time $a$ is
\[v(a)=\lim_{h \rightarrow 0}\frac{s(a+h)-s(a)}{h}\]
So, in our case:
\[v(1)=\lim_{h \rightarrow 0}\frac{s(1+h)-s(1)}{h} =
\lim_{h \rightarrow 0}\frac{[3(1+h)^2+5]-[3(1)^2+5]}{h} =
\lim_{h \rightarrow 0}\frac{6h+3h^2}{h}=
\lim_{h \rightarrow 0}6+3h = 6\]
So the velocity when $t=1$ is 6 units per second.
\end{enumerate}
\end{solution}


\begin{Mquestion}
Suppose the position of a body at time $t$ (measured in seconds) is given by
$s(t)=\sqrt{t}$.
\begin{enumerate}[(a)]
\item What is the average velocity of the object from $t=1$ second to $t=9$ seconds?
\item What is the velocity of the object at time $t=1$?
\item What is the velocity of the object at time $t=9$?
\end{enumerate}
\end{Mquestion}
\begin{hint}
Use that $\frac{\sqrt{a}-b}{c}
                  = \frac{\sqrt{a}-b}{c}\cdot
                            \left(\frac{\sqrt{a}+b}{\sqrt{a}+b}\right)
                  = \frac{a-b^2}{c(\sqrt{a}+b)}$.
 \end{hint}
\begin{answer}
\begin{enumerate}[(a)]
\item $\frac{1}{4}$ units per second
\item $\frac{1}{2}$ units per second
\item $\frac{1}{6}$ units per second
\end{enumerate}
Remark: the average velocity is \emph{not} the average of the two instantaneous velocities.
\end{answer}
\begin{solution}
\begin{enumerate}[(a)]
\item Average velocity: $\dfrac{\mbox{change in position}}{\mbox{change in time}} = \dfrac{s(9)-s(1)}{9-1} = \dfrac{3-1}{9-1}=\dfrac{1}{4}$ units per second.


\item From the notes, we know the velocity of an object at time $a$ is
\[v(a)=\lim_{h \rightarrow 0}\frac{s(a+h)-s(a)}{h}\]
So, in our case:
\begin{align*}
v(1)=\lim_{h \rightarrow 0}\frac{s(1+h)-s(1)}{h}
&=\lim_{h \rightarrow 0}\frac{\sqrt{1+h}-1}{h}\\
&=\lim_{h \rightarrow 0}\frac{\sqrt{1+h}-1}{h}\cdot \left( \dfrac{\sqrt{1+h}+1}{\sqrt{1+h}+1}\right)\\
&=\lim_{h \rightarrow 0}\frac{(1+h)-1}{h(\sqrt{1+h}+1)}\\
&=\lim_{h \rightarrow 0}\frac{1}{\sqrt{1+h}+1}=\frac{1}{2}\mbox{ units per second}
\end{align*}


\item \begin{align*}
v(9)=\lim_{h \rightarrow 0}\frac{s(9+h)-s(9)}{h}
&=\lim_{h \rightarrow 0}\frac{\sqrt{9+h}-3}{h}\\
&=\lim_{h \rightarrow 0}\frac{\sqrt{9+h}-3}{h}\cdot \left( \dfrac{\sqrt{9+h}+3}{\sqrt{9+h}+3}\right)\\
&=\lim_{h \rightarrow 0}\frac{(9+h)-9}{h(\sqrt{9+h}+3)}\\
&=\lim_{h \rightarrow 0}\frac{1}{\sqrt{9+h}+3}=\frac{1}{6}\mbox{ units per second}
\end{align*}
\end{enumerate}
Remark: the average velocity is \emph{not} the average of the two instantaneous velocities.
\end{solution}
