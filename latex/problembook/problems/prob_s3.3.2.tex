%
% Copyright 2018 Joel Feldman, Andrew Rechnitzer and Elyse Yeager.
% This work is licensed under a Creative Commons Attribution-NonCommercial-ShareAlike 4.0 International License.
% https://creativecommons.org/licenses/by-nc-sa/4.0/
%
\questionheader{ex:s3.3.2}


%%%%%%%%%%%%%%%%%%
\subsection*{\Conceptual}
%%%%%%%%%%%%%%%%%%

\begin{Mquestion}
Which of the following functions $T(t)$ satisfy the differential equation
$\ds\diff{T}{t}=5\left[T-20\right]$?
\[\mbox{(a) }
T(t)=20
\qquad
\mbox{(b) }
 T(t)=20e^{5t}-20
\qquad
\mbox{(c) }
 T(t)=e^{5t}+20
\qquad
\mbox{(d) }
 T(t)=20e^{5t}+20
\]
\end{Mquestion}
\begin{hint}
You can refer to Corollary~\ref*{cor:coolingDEsoln},
but you can also just differentiate the various proposed functions and see whether, in fact, $\ds\diff{T}{t}$ is the same as $5[T-20]$.
\end{hint}
\begin{answer} (a), (c), (d)
\end{answer}
\begin{solution}
Using Corollary~\ref*{cor:coolingDEsoln} (with $A=20$ and $K=5$),  solutions to the differential equation all have the form
\[T(t)=[T(0)-20]e^{5t}+20\] for some constant $T(0)$. This fits (a) (with $T(0)=20$), (c) (with $T(0)=21)$, and (d) (with $T(0)=40$), but not (b) (since the constant has the wrong sign).

Instead of using the corollary, we can also just check each function for ourselves.
\begin{itemize}
\item[(a)] $\ds\diff{T}{t}=0=5\cdot0=5[T(t)-20]$, so (a) gives a solution to the differential equation.
\item[(b)] $\ds\diff{T}{t}=5[20e^{5t}]=5[T+20] \neq 5[T-20]$, so (b) does not give a solution to the differential equation.
\item[(c)] $\ds\diff{T}{t}=5[e^{5t}]=5[T-20]$, so (c) gives a solution to the differential equation.
\item[(d)] $\ds\diff{T}{t}=5[20e^{5t}]=5[T-20]$,
so (d) gives a solution to the differential equation.
\end{itemize}
\end{solution}






\begin{question}
At time $t=0$, an object is placed in a room, of temperature $A$. After $t$ seconds, Newton's Law of Cooling gives the temperature of the object is as
\[T(t)=35e^{Kt}-10\]
What is the temperature of the room? Is the room warmer or colder than the object?
\end{question}
\begin{hint}
From Newton's Law of Cooling and Corollary~\ref*{cor:coolingDEsoln}, the temperature of the object will be
\[T(t)=[T(0)-A]e^{Kt}+A\]
where $A$ is the ambient temperature, $T(0)$ is the initial temperature of the copper, and $K$ is some constant.
\end{hint}
\begin{answer}
The temperature of the room is -10 degrees, and the room is colder than the object.
\end{answer}
\begin{solution}
From Newton's Law of Cooling and Corollary~\ref*{cor:coolingDEsoln}, the temperature of the object will be
\[T(t)=[T(0)-A]e^{Kt}+A\]
where $A$ is the ambient temperature (the temperature of the room), $T(0)$ is the initial temperature of the copper, and $K$ is some constant. So, the ambient temperature--the temperature of the room-- is $-10$ degrees. Since the coefficient of the exponential part of the function is positive, the temperature of the object is higher than the temperature of the room.
\end{solution}




\begin{Mquestion}\label{s3.3.2Knegative}
A warm object is placed in a cold room. The temperature of the object, over time, approaches the temperature of the room it is in. The temperature of the object at time $t$ is given by
\[T(t)=[T(0)-A]e^{Kt}+A.\]
Can $K$ be a positive number? Can $K$ be a negative number? Can $K$ be zero?
\end{Mquestion}
\begin{hint}
What is $\ds\lim_{t \to \infty}e^{Kt}$ when $K$ is positive, negative, or zero?
\end{hint}
\begin{answer} $K$ is a negative number. It cannot be positive or zero.
\end{answer}
\begin{solution}
As $t$ grows very large, $T(t)$ approaches $A$. That is:
\begin{align*}
\ds\lim_{t \to \infty}T(t)&=A\\
\lim_{t \to \infty}[T(0)-A]e^{Kt}+A&=A\\
\lim_{t \to \infty}[T(0)-A]e^{Kt}&=0
\intertext{Since the object is warmer than the room, $T(0)-A$ is a nonzero constant. So,}
\lim_{t \to \infty}e^{Kt}&=0
\end{align*}
This tells us that $K$ is a negative number. So, $K$ must be negative--not zero, and not positive.

Remark: in our work, we used the fact that the object and the room have different temperatures (but it didn't matter which one was hotter). If not, then $T(0)=A$, and $T(t)=A$: that is, the temperature of the object is constant. In this case, our usual form for the temperature of the object looks like this:
\[T(t)=0e^{Kt}+A\]
Keeping the exponential piece in there is overkill: the temperature isn't changing, the function is simply constant. If the object and the room have the same temperature, $K$ could be any real number since we're multiplying $e^{Kt}$ by zero.

Remark: contrast this to Question~\ref{s3.3.2Knegative2}.
\end{solution}


\begin{question}
Suppose an object obeys Newton's Law of Cooling, and its temperature is given by
\[T(t)=[T(0)-A]e^{kt}+A\]
for some constant $k$. At what time is $T(t)=A$?
\end{question}
\begin{hint}
Solve $A=[T(0)-A]e^{kt}$ for $t$.
\end{hint}
\begin{answer}
If the object has a different initial temperature than its surroundings, then $T(t)$ is \emph{never} equal to $A$. (But as time goes on, it gets closer and closer.)\\
If the object starts out with the same temperature as its surrounding, then $T(t)=A$ for all values of $t$.
\end{answer}
\begin{solution}
We want to know when
\begin{align*}
[T(0)-A]e^{kt}+A&=A
\intertext{That is, when}
[T(0)-A]e^{kt}&=0
\intertext{Since $e^{kt}>0$ for all values of $k$ and $t$, this happens exactly when}
T(0)-A=0
\end{align*}
So: if the initial temperature of the object is not the same as the ambient temperature, then according to this model, it never will be! (However, as $t$ gets larger and larger, $T(t)$ gets closer and closer to $A$--it just never exactly reaches there.)

If the initial temperature of the object starts out the same as the ambient temperature, then $T(t)=A$ for all values of $t$.
\end{solution}



%%%%%%%%%%%%%%%%%%
\subsection*{\Procedural}
%%%%%%%%%%%%%%%%%%


\begin{question}
A piece of copper at room temperature (25$^\circ$) is placed in a boiling pot of water. After 10 seconds, it has heated to 90$^\circ$. When will it be 99.9$^\circ$?
\end{question}
\begin{hint}
From Newton's Law of Cooling and Corollary~\ref*{cor:coolingDEsoln}, we know the temperature of the copper will be
\[T(t)=[T(0)-A]e^{Kt}+A\]
where $A$ is the ambient temperature, $T(0)$ is the initial temperature of the copper, and $K$ is some constant. Use the given information to find  an expression for $T(t)$ not involving any unknown constants.
\end{hint}
\begin{answer}
$\dfrac{-10\log(750)}{\log\left(\tfrac{2}{15}\right)}\approx 32.9$ seconds
\end{answer}
\begin{solution}
From Newton's Law of Cooling and Corollary~\ref*{cor:coolingDEsoln}, we know the temperature of the copper will be
\[T(t)=[T(0)-A]e^{Kt}+A\]
where $A$ is the ambient temperature (100$^\circ$), $T(0)$ is the temperature of the copper at time 0 (let's make this the instant it was dumped in the water, so $T(0)=25^\circ$), and $K$ is some constant. That is:
\begin{align*}
T(t)&=[25-100]e^{Kt}+100\\
&=-75e^{Kt}+100
\intertext{The information given tells us that $T(10)=90$, so}
90&=-75e^{10K}+100\\
75\left(e^K\right)^{10}&=10\\
\left(e^K\right)^{10}&=\frac{2}{15}\\
e^K&=\left(\frac{2}{15}\right)^{\tfrac{1}{10}}
\intertext{This lets us describe $T(t)$ without any unknown constants.}
T(t)&=-75\left(e^K\right)^{t}+100\\
&=-75\left(\frac{2}{15}\right)^{\tfrac{t}{10}}+100
\intertext{The question asks what value of $t$ gives $T(t)=99.9$.}
99.9&=-75\left(\frac{2}{15}\right)^{\tfrac{t}{10}}+100\\
75\left(\frac{2}{15}\right)^{\tfrac{t}{10}}&=0.1\\
\left(\frac{2}{15}\right)^{\tfrac{t}{10}}&=\frac{1}{750}\\
\log\left(\left(\frac{2}{15}\right)^{\tfrac{t}{10}}\right)&=\log\left(\frac{1}{750}\right)\\
\frac{t}{10}\log\left(\frac{2}{15}\right)&=-\log(750)\\
t&=\frac{-10\log(750)}{\log\left(\tfrac{2}{15}\right)}\approx 32.9
\end{align*}
It takes about 32.9 seconds.
\end{solution}






\begin{Mquestion}
Today is a chilly day. We heated up a stone to 500$^\circ$ C in a bonfire, then took it out and left it outside, where the temperature is 0$^\circ$ C. After 10 minutes outside of the bonfire, the stone had cooled to a still-untouchable 100$^\circ$ C. Now the stone is at a cozy 50$^\circ$ C. How long ago was the stone taken out of the fire?
\end{Mquestion}
\begin{hint}
From Newton's Law of Cooling and Corollary~\ref*{cor:coolingDEsoln}, we know the temperature of the stone $t$ minutes after it leaves the fire is
\[T(t)=[T(0)-A]e^{Kt}+A\]
where $A$ is the ambient temperature, $T(0)$ is the temperature of the stone the instant it left the fire, and $K$ is some constant.
\end{hint}
\begin{answer}
$10\dfrac{\log(10)}{\log(5)}\approx 14.3$ minutes
\end{answer}
\begin{solution}
The temperature of the stone $t$ minutes after taking it from the bonfire is
\begin{align*}
T(t)&=[T(0)-A]e^{Kt}+A\\
&=[500-0]e^{Kt}+0\\
&=500e^{Kt}
\intertext{for some constant $K$. We are given that $T(10)=100$.}
100=T(10)&=500e^{10K}\\
e^{10K}&=\frac{1}{5}\\
e^K&=5^{-\tfrac{1}{10}}
\intertext{This gives us the more complete picture for the temperature of the stone.}
T(t)&=500\left(e^K\right)^t=500\cdot 5^{-\tfrac{t}{10}}
\intertext{If $T(t)=50:$}
50=T(t)&=500\cdot 5^{-\tfrac{t}{10}}\\
\frac{1}{10}=10^{-1}&=5^{-\tfrac{t}{10}}\\
10&=5^{\tfrac{t}{10}}\\
\log(10)&=\frac{t}{10}\log(5)\\
t&=10\frac{\log(10)}{\log(5)}\approx 14.3
\end{align*}
So the stone  has been out of the fire for about 14.3 minutes.
\end{solution}





%%%%%%%%%%%%%%%%%%
\subsection*{\Application}
%%%%%%%%%%%%%%%%%%

\begin{question}[1997H]
Isaac Newton drinks his coffee with cream. To be exact, 9 parts coffee
to 1 part cream. His landlady pours him a cup of coffee at
$95^\circ$ C into which Newton stirs cream taken from the icebox at $5^\circ$ C.
When he drinks the mixture ten minutes later, he notes that it has cooled
to $54^\circ$ C.
 Newton wonders if his coffee would be hotter (and by how much) if he
waited until just before drinking it to add the cream. Analyze this question,
assuming that:
\begin{enumerate}[(i)]
\item The temperature of the dining room is constant at $22^\circ$ C.
\item When a volume $V_1$ of liquid at temperature $T_1$ is mixed
with a volume $V_2$ at temperature $T_2$, the temperature of the mixture
is $\dfrac{V_1T_1+V_2T_2}{V_1+V_2}$.
\item Newton's Law of Cooling: The temperature of an object
cools at a rate proportional to the difference in temperature between the
object and its surroundings.
\item The constant of proportionality is the same for the cup
of coffee with cream as for the cup of pure coffee.
\end{enumerate}
\end{question}
\begin{answer}
If Newton adds his cream just before drinking, the coffee
ends up {cooler by $0.85^\circ$ C}.
\end{answer}
\begin{solution}
\begin{itemize}
\item First scenario: At time $0$, Newton mixes 9 parts coffee at temperature
$95^\circ$ C with 1 part cream at temperature $5^\circ$ C. The resulting
mixture has temperature
$$
\frac{9\times 95+1\times 5}{9+1}=86^\circ
$$
The mixture cools according to Newton's Law of Cooling, with initial temperature 86$^\circ$ and ambient temperature 22$^\circ$:
\begin{align*}
T(t)&=[86-22]e^{-kt}+22\\
T(t)&=64e^{-kt}+22
\intertext{After 10 minutes,}
\textcolor{red}{54}=T(10)&=22+64e^{-10 k}\\
 e^{-10 k}&=\frac{54-22}{64}=\frac{1}{2}
\end{align*}
We could compute $k$ from this, but we don't need it.

\item Second scenario: At time $0$, Newton gets hot coffee at temperature
$95^\circ$ C. It cools according to Newton's Law of Cooling
\begin{align*}
T(t)=[T(0)-22]e^{-kt}+22
\end{align*}
In this second scenario, $T(0)=95$, so
$$
T(t)=[95-22]e^{-kt}+22=73e^{-kt}+22
$$
The value of $k$ is the same as in the first scenario, so after 10 minutes
$$
T(10)=22+73e^{-10k}=22+73\frac{1}{2}=58.5
$$
This cooled coffee is mixed with cold cream to yield a mixture of temperature
$$
\frac{9\times 58.5+1\times 5}{9+1}=\color{red}53.15
$$
\end{itemize}
Under the second (add cream just before drinking) scenario, the coffee
ends up {cooler by $0.85^\circ$ C}$\,$.
\end{solution}


\begin{Mquestion}[1997A]
 The temperature of a glass of iced tea is initially $5^\circ$.
After 5 minutes, the tea has heated to $10^\circ$ in a room where the air
temperature is $30^\circ$.
\begin{enumerate}[(a)]
\item Use Newton's law of cooling to obtain a differential equation
for the temperature $T(t)$ at time $t$.
\item Determine when the tea will reach a temperature of $20^\circ$.
\end{enumerate}
\end{Mquestion}
\begin{hint}
Newton's Law of Cooling models the temperature of the tea after  $t$ minutes as
\[T(t)=[T(0)-A]e^{Kt}+A\]
where $A$ is the ambient temperature, $T(0)$ is the initial temperature of the tea, and $K$ is some constant.
\end{hint}
\begin{answer}
\begin{enumerate}[(a)]
\item $\ds\diff{T}{t}=\frac{1}{5}\log\left(\frac{4}{5}\right)(T-30)$
\item $\dfrac{5\log(2/5)}{\log(4/5)}\approx 20.53$ min
\end{enumerate}
\end{answer}
\begin{solution}
\begin{enumerate}[(a)]
\item By Newton's law of cooling, the rate of change of temperature
is proportional to the difference between $T(t)$ and the ambient temperature,
which in this case is $30^\circ$. Thus
$$
\diff{T}{t}=k[T(t)-30]
$$
for some constant of proportionality $k$.
To answer part (a), all we have to do is find $k$.

Under Newton's Law of Cooling, the temperature at time $t$ will be given by
\begin{align*}
T(t)&=[T(0)-A]e^{kt}+A\\
&=[5-30]e^{kt}+30\\
&=-25e^{kt}+30
\intertext{We are told $T(5)=10$:}
10&=-25e^{5k}+30\\
25e^{5k}&=20\\
e^{5k}&=\frac{4}{5}\\
5k&=\log(4/5)\\
k&=\tfrac{1}{5}\log(4/5)
\intertext{So, the differential equation is}
\diff{T}{t}(t)&=\frac{1}{5}\log(4/5)[T(t)-30]
\end{align*}
\item
Since $T(t)=30-25e^{kt}$, the temperature of the tea is $20^\circ$ when
\begin{align*}
30-25 e^{kt}&=20\\
e^{kt}&=\frac{10}{25}\\
 kt&=\log\left(\frac{10}{25}\right)\\
 t&=\frac{1}{k}\log\frac{2}{5}\\
 &=\frac{5\log(2/5)}{\log(4/5)}\\
 &\approx{20.53 ~\mathrm{ min}}
\end{align*}\end{enumerate}
\end{solution}



\begin{Mquestion}\label{s3.3.2Knegative2}
Suppose an object is changing temperature according to Newton's Law of Cooling, and its temperature at time $t$ is given by
\[T(t)=0.8^{kt}+15\]
Is $k$ positive or negative?
\end{Mquestion}
\begin{hint}
What is $\ds\lim_{t \to \infty}T(t)$?
\end{hint}
\begin{answer}
positive
\end{answer}
\begin{solution}
As time goes on, temperatures that follow Newton's Law of Cooling get closer and closer to the ambient temperature. So, $\ds\lim_{t \to \infty} T(t)$ exists. In particular,
$\ds\lim_{t \to \infty} 0.8^{kt}$ exists.
\begin{itemize}
\item If $k<0$, then $\ds\lim_{t \to \infty}0.8^{kt} = \infty$, since $0.8<1$. So, $k \geq 0$.
\item If $k=0$, then $T(t)=16$ for all values of $t$. But, in the statement of the question, the object is changing temperature. So, $k>0$.
\end{itemize}
Therefore, $k$ is positive.

Remark: contrast this to Question~\ref{s3.3.2Knegative}. The reason we get a different answer is that our base in this question (0.8) is less than one, while the base in
Question~\ref{s3.3.2Knegative} ($e$) is greater than one.

Although the given equation $T(t)$ does not exactly look like the Newton's Law equations we're used to, it is equivalent.
Remembering
$e^{\log(0.8)}=0.8$:
\begin{align*}
            T(t) &= 0.8^{kt} + 15 \\
                 &= \big(e^{\log 0.8}\big)^{kt} + 15 \\
                 &= e^{(k\log 0.8)t} + 15 \\
                 &= [16-15]  e^{(k\log 0.8)t} + 15 \\
                 &= [16-15]  e^{Kt} + 15
       \end{align*}
       with $K=k\log 0.8$. This is now the more familiar form for Newton's Law of Cooling (with $A=15$ and $T(0)=16$).

 Since $0.8<1$, $\log(0.8)$ is negative, so $k$ and $K$ have opposite signs.
\end{solution}
